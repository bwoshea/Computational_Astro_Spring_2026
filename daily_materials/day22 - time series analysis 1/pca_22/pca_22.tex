\documentclass[10pt]{article}
 \usepackage[margin=1in]{geometry} 
\usepackage{amsmath,amsthm,amssymb,amsfonts,color, titling}
 \usepackage{listings}

\setlength{\droptitle}{-20mm} 

\usepackage[colorlinks=true,linkcolor=red,urlcolor=blue]{hyperref}

\title{Pre-class assignment \#22}
\author{PHY-905-005\\Computational Astrophysics and
  Astrostatistics\\Spring 2023}
 \date{} % leave blank to have no date

\begin{document}
 
\maketitle

\vspace{-10mm}

\noindent
\textbf{This assignment is due the evening of Monday April 24, 2023.}
 Turn in all materials via GitHub.

\vspace{5mm}

\noindent
\textbf{Reading:}

\begin{enumerate}

\item \textbf{Required:} Chapter 11, ``Time series analysis'' (pp. 292-336) of \textit{Modern Statistical
    Methods for Astronomy},
  by Feigelson \& Babu 

\item \textbf{Optional:} ``Understanding the
    Lomb-Scargle Periodogram'' by J. VanderPlas
    \href{https://arxiv.org/abs/1703.09824}{arXiv:1703.09824}, also
    \href{https://github.com/jakevdp/PracticalLombScargle/}{source code for paper}

\end{enumerate}

\noindent
\textbf{Your assignment:}

\begin{enumerate}

\item In the file \texttt{ANSWERS.md}, write down any questions that
you have about the material you read, any points that are not clear, or anything you'd
like to know more about.  Aim for at
least 3 questions/unclear points/etc. 

\item In the Jupyter notebook that accompanies this assignment, work
  through the exercises.

\end{enumerate}

\noindent \textbf{Handing it in:} Include your modified notebook and
your answers to the questions (in the file \texttt{ANSWERS.md})
in your assignment.

\end{document}
