\documentclass[10pt]{article}
 \usepackage[margin=1in]{geometry} 
\usepackage{amsmath,amsthm,amssymb,amsfonts,color, titling,graphicx}
 \usepackage{listings}

\setlength{\droptitle}{-20mm} 

\usepackage[colorlinks=true,linkcolor=red,urlcolor=blue]{hyperref}

\title{In-class assignment \#6}
\author{PHY-905-005\\Computational Astrophysics and
  Astrostatistics\\Spring 2023}
 \date{} % leave blank to have no date

\begin{document}
 
\maketitle

\vspace{-5mm}


\noindent \textbf{Instructions:}  We're going to build on the
pre-class assignment by working through GitHub's process for dealing
with the creation of issues and pull requests, as well as branching,
and forking.  \textbf{Get together in a group of three, and try to do
  the following:}

\begin{enumerate}

\item One member of the group should create a new git repository on
  GitHub, and give the other members access to it.  Put the
  \textbf{broken code} from the last in-class assignment in the
  repository and push it to GitHub.

\item Create two GitHub issues associated with this repository: one describing the horrific formatting
  of the repository, and one detailing the two bugs.  In general it's
  best to be as informative and factual as possible in an issue as a
  courtesy to the developer, even pointing out exact lines of code, etc.
  
\item The other group member(s) should clone the repository to their
  own machines.  One of you should \href{https://git-scm.com/book/en/v2/Git-Branching-Basic-Branching-and-Merging}{make a branch} called
  \texttt{cleaned-code}, \href{https://git-scm.com/docs/git-checkout}{check out} that branch (which you can
  verify with ``\texttt{git status}''), run \href{https://www.pylint.org/}{\texttt{pylint}} on the code in that
  branch, commit your changes, and push them back to github (don't
  forget to push the branch -- see the
  \href{https://github.com/git-guides/git-push}{GitHub docs} for instructions).  Another
  group member should make a second branch called \texttt{fixed-code},
  check out that branch,
  fix the two code bugs from last time, and push those changes and new
  branch to
  GitHub as well.
  Coordinate with your group members to make sure that you've edited
  at least one of the same lines of code in a different way in each
  branch -- \textbf{you want to deliberately create a merge conflict
    for a later step in this assignment!}

\item Now, each of the two people who created a new branch should
  issue a pull request (PR) \textbf{from their branch to the main branch}
  (so there will be two pull requests).  Make sure to describe what
  you've done, and \textbf{refer to the issue describing the problem
    by its number} (if you do it correctly it will automatically
  create a link). Don't do anything with this PR 
  yet!

\item Then, the person/people who did NOT create that pull request
  should comment on the PR with one or more suggested
  changes (you can make general comments or comments associated with
  specific lines of code -- try doing both!).  The creator of the PR should update the
  branch on their own machine and push to GitHub, which should update
  the GitHub PR.  The original author of the PR should respond to the comments.  If the commentors feel
  that you have done things correctly, then they can approve the pull
  request on GitHub (look under the ``Files Changed'' tab for the
  ``Review changes'' button to approve).  Once both commentors approve, \textbf{merge only one of
    the branches!}

 \item Assuming your second (unmerged) branch also has a line of code
   edited that was also modified in the merged PR, you will now have a
   \href{https://docs.github.com/en/pull-requests/collaborating-with-pull-requests/addressing-merge-conflicts/about-merge-conflicts}{merge
     conflict}.  If it's small you can resolve it
   \href{https://docs.github.com/en/pull-requests/collaborating-with-pull-requests/addressing-merge-conflicts/resolving-a-merge-conflict-on-github}{via
   the GitHub web interface}, but if it's complicated you will have to
 do it
 \href{https://docs.github.com/en/pull-requests/collaborating-with-pull-requests/addressing-merge-conflicts/resolving-a-merge-conflict-using-the-command-line}{via
 the command line using a text editor}.  \textbf{Fix the merge
 conflict, get two approvals on the PR, and merge!}

\item Now that your two issues have been address, make sure to close
  the issues.  It's good practice to make a final comment saying
  something to the effect of ``Fixed in PR\#1234.  Closing issue.''
  This provides a historical record that the issue was actually fixed,
  and where!

\item You no longer need the two branches that you've created, so
  \href{https://stackoverflow.com/questions/2003505/how-do-i-delete-a-git-branch-locally-and-remotely}{delete
    them both locally on your laptop and on GitHub}.  

\end{enumerate}



%\vspace{5mm}
\noindent
\textbf{Congratulations, you're all done!}  If you've finished the
process described above and still have some time left in class, try
replicating it but fix the changes by \textbf{forking the code on
GitHub rather than making branches}.  This will create entirely new
repositories instead of just adding a branch to the existing one, but
otherwise it's largely the same workflow.


\end{document}