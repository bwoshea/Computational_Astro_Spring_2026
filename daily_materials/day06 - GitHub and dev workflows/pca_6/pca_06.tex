\documentclass[10pt]{article}
 \usepackage[margin=1in]{geometry} 
\usepackage{amsmath,amsthm,amssymb,amsfonts,color, titling}
 \usepackage{listings}

\setlength{\droptitle}{-20mm} 

\usepackage[colorlinks=true,linkcolor=red,urlcolor=blue]{hyperref}

\title{Pre-class assignment \#6}
\author{PHY-905-005\\Computational Astrophysics and
  Astrostatistics\\Spring 2023}
 \date{} % leave blank to have no date

\begin{document}
 
\maketitle

\vspace{-5mm}

\noindent
\textbf{This assignment is due the evening of Monday January 30, 2023.}
 Turn in all materials via the GitHub Classroom.

\vspace{5mm}

\noindent
\textbf{The overall purpose of this pre-class assignment} is to help
you think through the development of open source scientific software
-- in particular, the techniques and tools that are considered to be
``best practices'' by researchers who develop software that is likely
to be used for a long time, and possibly by other researchers.  While
you may not use all of these tools and practices right now, it's
important for you to know about them, and to have some familiarity
with their use for when it becomes relevant!

Note that there are quite a few readings in this pre-class assignment.
Most are quite short, but for the longer ones I want you to read
quickly and try to take away the key points.  And, as you're reading,
make notes on those key points and also about how you might use the
things you're reading about in your own work!

\vspace{5mm}

\noindent
\textbf{Readings about software development and workflows:}

\begin{enumerate}

\item Look at the E-CAM software library's
  \href{https://e-cam.readthedocs.io/en/latest/best-practices/index.html#best-practices}{Scientific
  Software Best Practices} -- in particular, the ``General Programming
Guidelines'' section.  \textbf{Read this carefully!}

\item Revisit the Git tutorial (which you looked at prior to the first
  day of class):
  \url{http://rogerdudler.github.io/git-guide/}

\item
  \href{https://www.pluralsight.com/blog/software-development/the-definitive-guide-to-forks-and-branches-in-git}{The
  Definitive Guide to Forks and Branches in Git}  (blog post)

\item \href{https://docs.github.com/en/pull-requests}{GitHub
    documentation on Pull Requests} -- make sure to read the ``Getting Started,''
  ``Create \& Delete Branches,'' ``About Pull Requests,'' ``Creating a
  pull request,''  ``Creating a PR from a fork,'' ``About merge
  conflicts,'' and ``Resolve merge conflicts'' sub-pages.  Browse the
  rest if you have time.

\item \href{https://docs.github.com/en/issues}{GitHub documentation on
  issues} -- make sure to read the ``About issues,'' ``Create an
Issue,'' ``Link PR to issue,'' and ``Assign issues \& PRs''
sub-pages.  Browse the rest if you have time.

\end{enumerate}


\vspace{5mm}

\noindent
\textbf{Reading about software testing:}

\begin{enumerate}

\item
  \href{https://www.atlassian.com/continuous-delivery/software-testing/types-of-software-testing}{Types
  of software testing}.  

\item
  \href{https://en.wikipedia.org/wiki/Continuous_integration}{Wikipedia
  article on continuous integration}  (as a software development process)


\item \href{https://docs.github.com/en/actions}{GitHub Actions documentation}

\item \href{https://docs.pytest.org/en/7.2.x/}{PyTest documentation}

\item \href{https://docs.python.org/3/library/unittest.html}{unittest}
  Python unit testing framework  (reference -- look at this in
  comparison to PyTest)

  \end{enumerate}


  \vspace{5mm}

\noindent
\textbf{Reading about software style, linting, and static analysis:}

\begin{enumerate}

\item \href{https://peps.python.org/pep-0008/}{Python PEP 8 Style
    Guide} (reference - only skim this to get the main point!)

\item \href{https://pypi.org/project/pycodestyle/}{Pycodestyle documentation}
  
\item \href{https://en.wikipedia.org/wiki/Lint_(software)}{Wikipedia
    article on software linting}

\item
  \href{https://en.wikipedia.org/wiki/Static_program_analysis}{Wikipedia
  article on static program analysis}

\item \href{https://pylint.readthedocs.io/en/latest/index.html}{Pylint
  documentation}

 \end{enumerate}


\vspace{5mm}

\noindent
\textbf{Your assignment:}  
%\vspace{3mm}

\begin{enumerate}

  \item For each of the sets of readings, summarize your key
    takeaways and remaining questions in \texttt{ANSWERS.md}.  What
    points that were made resonate with you?  What tools seem like
    they might be particularly useful?
  
  \item Make sure that you can do the following things with git and
  GitHub, by working through the Git tutorial and GitHub documents
  linked to above.  In \texttt{ANSWERS.md}, make notes about what went well and
  what you struggled with.  

  \begin{enumerate}

  \item Create an empty repository on GitHub.
  \item Clone it to your local computer.
  \item Add some code to it (from a previous pre- or in-class
    assignment), commit the new code, and push the changes to GitHub.
  \item Make a local branch (on your local computer), make changes to
    the code in
    that branch, and push those changes to GitHub.
  \item Merge the changes from your branch into your master branch and
    delete the old branch.  Push those changes to GitHub.
  \end{enumerate}



  \item Consider the code that you used for the in-class assignment on
    debugging.  What kind of testing might you implement to make sure
    that the bugs you found don't occur in that code again?  How might
    you go about using either PyTest or unittest?  How might you
    create an integration test for this code?

  \item Find a relatively complicated piece of Python code that you've
    written in the past and run both \texttt{Pycodestyle} and
    \texttt{Pylint} on it.  If you are using the Anaconda python
    distribution these should already be installed; if not, you will
    have to install them at the command line with ``\texttt{pip
      install pycodestyle}'' and ``\texttt{pip install pylint}'',
    respetively.  Note that you cannot always run these on Jupyter Notebooks
    -- you should try running it on standalone Pythone code.  Make
    notes in \texttt{ANSWERS.md} about the similarities and
    differences between the two tools.

  \item For the piece of code described above, what do you have to do
    in order to get rid of all of the errors produced by
    \texttt{Pylint}?  Does it seem to produce better code?
    (And what does ``better'' mean to you?)

  \item Finally, \textbf{pick one} of the following astrophysics community
    codes, look at its GitHub project page, and try to identify which of the
    best practices described in the readings are (or aren't!)
    implemented.  Make sure to look at the Issues, Pull Requests, and
    Actions tabs for the one you've chosen and see what's going on
    there.  Also look at the documentation and see what seems
    particularly useful to a new user!
    Make some notes in
    \texttt{ANSWERS.md} about your findings.

    \begin{enumerate}
      \item \href{https://github.com/yt-project/yt}{YT GitHub}
        (\href{https://yt-project.org/}{Project web page};
        \href{http://yt-project.org/docs/dev/}{documentation}).  YT is
        a widely-used data analysis and visualization tool.
      \item \href{https://github.com/enzo-project/enzo-dev}{Enzo GitHub}
        (\href{https://enzo-project.org/}{Project web page};
        \href{https://enzo.readthedocs.io/en/latest/}{documentation}).
        Enzo is a multiphysics self-gravitating fluid dynamics code
        that is primarily used for cosmological structure formation
        and star formation.  
      \item \href{https://github.com/tardis-sn/tardis/}{Tardis
          GitHub}.  Tardis is a a radiative transfer spectral
        synthesis code used to model supernova ejecta.
        (\href{https://tardis-sn.github.io/tardis/}{Project web page}; \href{https://tardis-sn.github.io/tardis/}{documentation})
      \item
        \href{https://github.com/parthenon-hpc-lab/parthenon}{Parthenon
        GitHub}
      (\href{https://github.com/parthenon-hpc-lab/parthenon/tree/develop/docs}{documentation}).
      Parthenon is a GPU-optimized adaptive mesh framework based off of the
      Athena++ code, which serves as the basis for several very fast
      and scalable plasma astrophysics and numerical relativity codes.
    \end{enumerate}

\end{enumerate}


 
\vspace{5mm}

\noindent \textbf{Handing it in:} Include your modified code and
your answers to the questions (in the file \texttt{ANSWERS.md})
in your assignment.



\end{document}
