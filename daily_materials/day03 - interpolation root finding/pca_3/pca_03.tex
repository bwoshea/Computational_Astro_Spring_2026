\documentclass[10pt]{article}
 \usepackage[margin=1in]{geometry} 
\usepackage{amsmath,amsthm,amssymb,amsfonts,color, titling}
 \usepackage{listings}

\setlength{\droptitle}{-20mm} 

\usepackage[colorlinks=true,linkcolor=red,urlcolor=blue]{hyperref}

\title{Pre-class assignment \# 3}
\author{PHY-905-005\\Computational Astrophysics and
  Astrostatistics\\Spring 2023}
 \date{} % leave blank to have no date

\begin{document}
 
\maketitle

\vspace{-5mm}

\noindent
\textbf{This assignment is due the evening of Wednesday, January 18,
  2023.}

\vspace{5mm}

\noindent
\textbf{Instructions:} Read the materials below and follow
the instructions/answer the questions at the end.  Turn in your code
and all materials via the GitHub Classroom.

\vspace{5mm}

\noindent
\textbf{What to turn in:} Turn in plots (and the scripts
used to generate them), source code, etc.
\textbf{Do not} turn in object files, binary files, or very large
files of any kind unless you are explicitly asked to do so!

\vspace{5mm}

\noindent
\textbf{Reading:}

\begin{enumerate}

\item Interpolation of data: Sections 2.1 and 2.4 of \textit{An Introduction to Computational Physics},
  by T. Pang (PDF; provided).  

\item Root and extrema finding:  Sections 3.3 and 3.4 of \textit{An Introduction to Computational Physics},
  by T. Pang (PDF; provided) 

\item 
  \href{https://en.wikipedia.org/wiki/Interpolation}{Wikipedia article
    on interpolation techniques}  (optional, but highly recommended)

\item 
  \href{https://en.wikipedia.org/wiki/Root-finding_algorithm}{Wikipedia article
    on root-finding techniques}  (optional, but highly recommended)

\item Interpolation: Section 5.11 of \textit{Computational Physics}, by M. Newman (optional)

\item Root and extrema finding:  Sections 6.3 and 6.4 of of
  \textit{Computational Physics}, by M. Newman (optional)

\end{enumerate}

\vspace{5 mm}

\noindent
\textbf{Questions:}

\begin{enumerate}

\item Implement the linear interpolation method as a function and test it on the
  arrays of x and $f(x)$ values generated by the file
  \texttt{interp.py} for values of x
  that are midway between the given data points.  Compare the outputs
  of your linear interpolation to the actual analytic function (also
  given in \texttt{interp.py}).  How good of a job does it do?  Now,
  try doing this for the same values of x using the \href{https://docs.scipy.org/doc/scipy/reference/}{SciPy}
\href{https://docs.scipy.org/doc/scipy/reference/interpolate.html}{\texttt{scipy.interpolate}}
package -- in particular,
\href{https://docs.scipy.org/doc/scipy/reference/generated/scipy.interpolate.interp1d.html}{\texttt{interp1d}}.
\texttt{interp1d}  lets you try different methods using the
\texttt{kind} argument.  Try the \texttt{linear},
\texttt{nearest}, and \texttt{cubic} methods.  How well do these perform compared to the
routine you've written?

\item Implement the bisection method and Newton's method for finding
  the roots of a mathematical function.   Put each one into its own
  function with any arguments that are necessary.  Test the two methods on the three functions 
  in the file \texttt{root.py}, using guesses for your starting point
  that are both close to and far from the actual root.  Some of these
  functions are better behaved than others for different root-finding
  methods; print out the values of $x$ and $f(x)$ (and $f'(x)$ if
  relevant) as your code iterates.  Describe the outcome in \texttt{ANSWERS.md}.

\item After you have done the required reading and implemented the
  code described above, what remaining questions do you have about
  these numerical techniques?  List those in \texttt{ANSWERS.md}.

\end{enumerate}



\end{document}