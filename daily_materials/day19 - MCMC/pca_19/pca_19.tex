\documentclass[10pt]{article}
 \usepackage[margin=1in]{geometry} 
\usepackage{amsmath,amsthm,amssymb,amsfonts,color, titling}
 \usepackage{listings}

\setlength{\droptitle}{-20mm} 

\usepackage[colorlinks=true,linkcolor=red,urlcolor=blue]{hyperref}

\title{Pre-class assignment \#19}
\author{PHY-905-005\\Computational Astrophysics and
  Astrostatistics\\Spring 2023}
 \date{} % leave blank to have no date

\begin{document}
 
\maketitle

\vspace{-10mm}

\noindent
\textbf{This assignment is due the evening of Wednesday April 12, 2023.}
 Turn in all materials via GitHub.

\vspace{5mm}

\noindent
\textbf{Reading:}

\begin{enumerate}

\item Wikipedia article on
  \href{https://en.wikipedia.org/wiki/Markov\_chain\_Monte\_Carlo}{Markov
    Chain Monte Carlo} techniques.

\item Wikipedia article on the
  \href{https://en.wikipedia.org/wiki/Metropolis\%E2\%80\%93Hastings_algorithm}{Metropolos-Hastings
    Algorithm}.

\item Tutorial on
  \href{https://sciencehouse.wordpress.com/2010/06/23/mcmc-and-fitting-models-to-data/}{MCMC
  and fitting models to data}.

\item \href{http://www.mcmchandbook.net/HandbookChapter1.pdf}{Chapter 1} of the
  \href{http://www.mcmchandbook.net/}{\textit{The MCMC Handbook}} by
  Brooks et al.  (PDF included; Optional, but highly recommended).

\item Reference for later:  Annual Review article on ``Markov Chain Monte Carlo Methods for
  Bayesian Data Analysis in Astronomy,'' by Sharma (\href{https://www.annualreviews.org/doi/10.1146/annurev-astro-082214-122339}{Publisher
    version}; \href{https://arxiv.org/abs/1706.01629}{arXiv version})

\item Reference for later:  ``Data Analysis Recipes: Using Markov
  Chain Monte Carlo'' by Hogg \& Foreman-Mackey
  (\href{https://iopscience.iop.org/article/10.3847/1538-4365/aab76e}{2018,
  ApJS, 236, 11})

\end{enumerate}

\noindent
\textbf{Your assignment:}

\begin{enumerate}

\item Work through the tasks in the included Jupyter notebook, and
answer the questions at the bottom of the notebook.  Make sure to work
through the MCMC code and see how it behaves!

\item In the file \texttt{ANSWERS.md}, write down any questions that
you have about the material you read or the work you did in the
Jupyter notebook, any points that are not clear, or anything you'd
like to know more about.  Aim for at
least 3 questions/unclear points/etc. 

\end{enumerate}

\noindent \textbf{Handing it in:} Include your modified notebook and
your answers to the questions (in the file \texttt{ANSWERS.md})
in your assignment.

\vspace{5mm}

\noindent
\textbf{Some open-source MCMC Python packages:}  This will probably be useful
for you in the future!

\begin{enumerate}
\item \href{https://emcee.readthedocs.io/en/stable/}{emcee - the MCMC
    hammer}.  This package is widely used in astronomy.

\item \href{https://www.pymc.io/}{PyMC} - a package that
  implements Bayesian statistical models and fitting tools, including MCMC.

\item \href{https://bmcmc.readthedocs.io/en/latest/}{bmcmc} - a
  general-purpose Bayesian MCMC package  (\href{https://github.com/sanjibs/bmcmc/}{GitHub repository}).
  Note: this is less mature than emcee and PyMC, but by the author of
  the Annual Reviews article linked to above.

\end{enumerate}

\end{document}
