\documentclass[10pt]{article}
 \usepackage[margin=1in]{geometry} 
\usepackage{amsmath,amsthm,amssymb,amsfonts,color, titling}
 \usepackage{listings}

\setlength{\droptitle}{-20mm} 

\usepackage[colorlinks=true,linkcolor=red,urlcolor=blue]{hyperref}

\title{In-class assignment \#7}
\author{PHY-905-005\\Computational Astrophysics and
  Astrostatistics\\Spring 2023}
 \date{} % leave blank to have no date

\begin{document}
 
\maketitle

\vspace{-5mm}

\noindent \textbf{Instructions:}   We're going to use the code for
your pre-class assignment, where you modeled the behavior of a mass on
a spring over some interval of time.  You've already implemented the Euler
method and a simple predictor-corrector method.  Now, implement (1)
either the
  \href{https://en.wikipedia.org/wiki/Semi-implicit_Euler_method}{Euler-Cromer
  method}  or the 
  \href{https://en.wikipedia.org/wiki/Midpoint_method}{\textit{explicit}
    midpoint
    method} and (2) the
  \href{https://en.wikipedia.org/wiki/Runge%E2%80%93Kutta_methods}{4th
                                order Runge-Kutta method} (RK4), and answer
                              the following questions:

\begin{itemize}

\item For given timestep sizes, $\Delta t = 0.1 \pi, 0.01 \pi,$ and
  $0.001 \pi$, what is the
  difference in the relative change in energy between the Euler
  method and these two methods?

\item Assume you wish to maintain energy to a given level of accuracy
  - say 0.01\% between $t=0$ and $t=4\pi$.  How many time steps of
  each of your methods do you need to reach that level of accuracy?
  How many total floating-point operations is that for each method for
  the entire integration?   To attain a given
  level of accuracy, which is the least computationally expensive method to use?

\item If you wanted to maintain energy to a given level of accuracy
  from time step to time step, how might you go about devising an
  algorithm to do it?  In other words, how to you change the way that
  you do integration so that you can adapt your time steps to save
  computational cost?

\end{itemize}

\noindent
Make some notes in \texttt{ANSWERS.md}, and
we'll also discuss it in class.

\vspace{5mm}

\noindent \textbf{Some hints/suggestions:}

\begin{itemize}

  \item A Python implementation of the 4th order Runge-Kutta method is in
    Section 8.1.3 of Newman -- I suggest you adapt that rather than
    trying to create it from scratch!

  \item Make sure that you use the exact same times for each algorithm
    when plotting,
    otherwise you'll see ``phase drift'' in your errors!

   \item Write code as functions to the greatest extent possible --
     you'll be able to reuse it later in the class!
   
\end{itemize}


\vspace{5mm}

\noindent 
\textbf{What to turn in:} Turn in \texttt{ANSWERS.md}, any
source code you wrote, any plots you created (and the scripts you used
to create them).  \textbf{Do not} turn in object files or
executables!

\end{document}