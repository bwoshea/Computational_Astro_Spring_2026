\documentclass[10pt]{article}
 \usepackage[margin=1in]{geometry} 
\usepackage{amsmath,amsthm,amssymb,amsfonts,color, titling}
 \usepackage{listings}

\setlength{\droptitle}{-20mm} 

\usepackage[colorlinks=true,linkcolor=red,urlcolor=blue]{hyperref}

\title{Pre-class assignment \#7}
\author{PHY-905-005\\Computational Astrophysics and
  Astrostatistics\\Spring 2023}
 \date{} % leave blank to have no date

\begin{document}
 
\maketitle

\vspace{-5mm}

\noindent
\textbf{This assignment is due the evening of Wednesday, February 1, 2023.}
 Turn in your code
and all materials via the GitHub Classroom.

\vspace{5mm}

\noindent
\textbf{Reading:}

\begin{enumerate}

\item Methods of solving initial value ODEs: Sections 8.1-8.5 of
  \textit{Computational Physics}, by Newman.   Note: you can skim
  Sections 8.5.2-8.5.6, but try to get the general idea of why these
  various methods exist!

\item 
  \href{https://en.wikipedia.org/wiki/Semi-implicit_Euler_method}{Wikipedia article
    on the Euler-Cromer method}  

\item 
  \href{https://en.wikipedia.org/wiki/Midpoint_method}{Wikipedia article
    on the midpoint method} 

\item (Optional) Methods of solving initial value ODEs: Sections 4.1-4.4  of \textit{An Introduction to Computational Physics},
  by T. Pang (PDF provided)


\end{enumerate}

%\vspace{5 mm}

\noindent
\textbf{Your assignment:}
Consider the simple mass on a spring, acting in the absence of gravity
so that the only force acting on it is a linear restoring force
pointing toward its resting position:

\begin{equation}
F = -k x
\end{equation}

where F is force, k is a spring constant, and x is position.  The
total energy of the spring is:

\begin{equation}
E = \frac{1}{2} k x^2 + \frac{1}{2}  m v^2
\end{equation}

where m is the mass of the object at the end of the spring and v is
its velocity.  Assume that at $t=0$ the spring is at its resting
position ($x=0$) with $v=1$.  The object on the spring as $m=1$ and
the spring constant is $k=1$.  Analytically, we can easily derive
the position and velocity of the object on the spring as a function of
time:

\begin{equation}
x(t) = x_m \sin(\omega t)
\end{equation}

\begin{equation}
v(t) = x_m \omega \cos(\omega t)
\end{equation}

with $\omega = (k/m)^{1/2} = 1$ and $x_m = 1$.
Integrate the ODEs that describe this system from $t=0$ to $t=4\pi$
using two different methods -- the Euler method and the
leapfrog method (described in Sections 8.1 and 8.5.1 of Newman,
respectively).  Use functions as appropriate, and also use the class
coding standard!
Use fixed timesteps of $\Delta t = 0.1 \pi$, $0.01 \pi$, and
$0.001 \pi$.  Using \textbf{one graph per method}, plot the position as a
function of time for the analytic solution and each of the three time
steps, using different colors or symbols so you can clearly
differentiate the time steps and adding a legend as appropriate.  Then, \textbf{on a single graph}, show
how well energy is conserved by both methods for the three choices of
time step between the starting time and $t = 4 \pi$.  You can do this
by making a plot of relative change in energy, $\epsilon$, between the beginning ($e_0$)
and end ($e_f$) of the simulation: 

\begin{equation}
\epsilon =
\frac{|e_f - e_0|}{e_0}
\end{equation}

\textbf{Make sure to double-check that you're comparing the same times!}  If you
don't take the same timesteps, and end up with the same actual time
(due to accumulation of floating-point error) you might have slightly
different answers.

Do you see a difference in behavior between the two methods?  Record
your observations, as well as any questions you have after doing the
readings and this assignment, in the file \texttt{ANSWERS.md}.

\end{document}
