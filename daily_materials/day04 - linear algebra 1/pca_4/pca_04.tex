\documentclass[10pt]{article}
 \usepackage[margin=1in]{geometry} 
\usepackage{amsmath,amsthm,amssymb,amsfonts,color, titling}
 \usepackage{listings}

\setlength{\droptitle}{-20mm} 

\usepackage[colorlinks=true,linkcolor=red,urlcolor=blue]{hyperref}

\title{Pre-class assignment \# 4}
\author{PHY-905-005\\Computational Astrophysics and
  Astrostatistics\\Spring 2023}
 \date{} % leave blank to have no date

\begin{document}
 
\maketitle

\vspace{-5mm}

\noindent
\textbf{This assignment is due the evening of Monday January 23, 2023.}
 Turn in your code
and all materials via the GitHub Classroom.

\vspace{5mm}

\noindent
\textbf{Reading:}

\begin{enumerate}

\item Section 6.1 of \textit{Computational Physics}, by M. Newman 

\item 
  \href{https://en.wikipedia.org/wiki/Gaussian_elimination}{Wikipedia article
    on Gaussian elimination} 

\item Numerical methods for matrices, part 1: Sections 5.1-5.3  of \textit{An Introduction to Computational Physics},
  by T. Pang (optional; PDF provided)

\item 
  \href{https://docs.scipy.org/doc/numpy/reference/routines.linalg.html}{NumPy
  linear algebra routines}  (reference)

\item 
  \href{https://docs.scipy.org/doc/scipy/reference/linalg.html}{SciPy
  linear algebra routines}  (reference)

\end{enumerate}

\noindent
\textbf{Your assignment:}  
\vspace{3mm}

For the matrices A and B in the included
file \texttt{matrix\_1.py}, write two functions that uses Gaussian
elimination with partial pivoting to (1) get the upper-triangular
matrix and (2) calculate the determinant of the matrix using the
upper-triangular matrix.  Do this without destroying the original
matrix by making a copy of the matrix (with \texttt{numpy.copy}) and
compare your calculated determinant to the output of the
\href{https://docs.scipy.org/doc/scipy/reference/generated/scipy.linalg.det.html#scipy.linalg.det}{\texttt{scipy.linalg.det}}
  routine, as run on the same matrix.  Suggestion: invent your own
  small matrix (perhaps a $3 \times 3$ matrix) to test your routines on so that
  you can verify that it's behaving correctly on a step-by-step basis!
  \textbf{Make sure to write your code as functions that take input
    matrices -- you're going to need those functions in class!}


Submit your \textit{easy-to-read and commented} code to the
repository (use the \texttt{class\_coding\_standard.md} document as a
basis for this, which is included in this repository).  Also, put any remaining questions that you might have 
 in the file \texttt{ANSWERS.md}, and push all of the files as your
 homework submission!

\end{document}
