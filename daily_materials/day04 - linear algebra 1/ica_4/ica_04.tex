\documentclass[10pt]{article}
 \usepackage[margin=1in]{geometry} 
\usepackage{amsmath,amsthm,amssymb,amsfonts,color, titling,graphicx}
 \usepackage{listings}

\setlength{\droptitle}{-20mm} 

\usepackage[colorlinks=true,linkcolor=red,urlcolor=blue]{hyperref}

\title{In-class assignment \#4}
\author{PHY-905-005\\Computational Astrophysics and
  Astrostatistics\\Spring 2023}
 \date{} % leave blank to have no date

\begin{document}
 
\maketitle

\vspace{-5mm}


\noindent \textbf{Instructions:}   Consider the circuit diagram shown
below.  Assuming $V_0 = 100$~V, $R_0 = 100~\Omega$, $R1 = R2 =
50~\Omega$, and $R_3 = 75~\Omega$, use Kirchoff's current and voltage
laws to create a set of linear equations that you can solve
numerically to determine the currents going through each resistor.  Modify the Gaussian Elimination code you wrote for your
pre-class assignment to solve a system of linear equations using
backward substitution, and verify analytically that you have obtained
the correct answer.  Also verify it using the appropriate SciPy linear
algebra routine, which should be very easy to do.  Upload your code when you're done!

\vspace{10mm}

\begin{figure}[h]
  \centering
    \includegraphics[width=0.75\textwidth]{circuit.pdf}
  \caption{Simple circuit diagram.  Totally not an image of something
  I wrote on a piece of paper and then took a picture of with my phone.}
\end{figure}

\vspace{5mm}
\noindent
\textbf{If you finish the problem listed above relatively quickly,}
try to implement the algorithm for LU decomposition described in
Section 6.1.4 of Newman so that you can calculate currents for an
arbitrary input voltage, $V_0$, and verify analytically that this
works.  Upload this code as well, in a separate file!

\end{document}