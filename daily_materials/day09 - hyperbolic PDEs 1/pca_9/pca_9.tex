\documentclass[10pt]{article}
 \usepackage[margin=1in]{geometry} 
\usepackage{amsmath,amsthm,amssymb,amsfonts,color, titling}
 \usepackage{listings}

\setlength{\droptitle}{-20mm} 

\usepackage[colorlinks=true,linkcolor=red,urlcolor=blue]{hyperref}

\title{Pre-class assignment \#9}
\author{Brian O'Shea, \\PHY-905-003, Computational Astrophysics and
  Astrostatistics\\Spring 2023}
 \date{} % leave blank to have no date

\begin{document}
 
\maketitle

\vspace{-5mm}

\noindent
\textbf{This assignment is due the evening of Wednesday Feb. 8, 2023.}
 Turn in all materials via GitHub.

\vspace{5mm}

\noindent
\textbf{Reading:}

\begin{enumerate}

\item Chapter 4 (``Advection Basics'') of Mike Zingale's
  \href{http://bender.astro.sunysb.edu/hydro_by_example/CompHydroTutorial.pdf}{Computational
  Hydrodynamics Tutorial} (also included in this repository).

\end{enumerate}

\noindent
\textbf{Your assignment:}  

\vspace{5mm}

You're going to solve the linear advection equation (Equation 4.1) for
``top hat'' and Gaussian wave pulses in a 1D domain, and experiment with two
different first-order methods of solution -- upwinding and
forward-time, centered-space (FTCS) -- and compare how they behave.
We will do this by creating a set of functions that do the following
(and which we will re-use both in class and later in the semester):

\begin{enumerate}

\item A function that takes in several arguments and generates a set
of initial conditions comprised of 1-D numpy arrays.  The input
arguments should be the number of points in the domain, $N$, and the
type of initial condition (``top hat'' wave or Gaussian wave).  The
function will \textbf{generate and return} a 1D array of point
locations in the domain $x=0$ to $x=1$, assuming N equidistant points,
as well as an array of $a$ values.  For a top hat wave, set $a=1.0$
for $0.35 < x < 0.65$ and $a=0.0$ outside of that, and for a Gaussian
wave set a max value of $a=1.0$ at $x=0.5$ and a full-width, half-max
of $0.1$.  You should make both arrays the same length (N
elements).

\item A function that evolves the system one time step (from step $n$
to $n+1$).  The input arguments should be the $a$ array described
above (at time $n$), the wave speed $u$, the CFL number $C$, and the
method that you wish to use (either upwind or FTCS).  The returned
value should be a new array corresponding to the $a$ array at time
$n+1$.  Assume that the domain is periodic -- in other words, think of
your $a$ array as looping around such that $a[N-1]$ is next to $a[0]$.

\item A function to make a plot of the simulation at some state,
  compared to the $t=0$ dataset, and save it to a
file with a unique name.  You should take as arguments the $x$ and $a$
arrays, plus whatever else you need to generate a unique file name and
title.
   
\end{enumerate}

\noindent
You may find the numpy array and Python function tutorial notebooks 
included in this Git repository to be helpful in constructing your
functions!

\vspace{2mm}
Using the functions you created above, evolve at top-hat wave with a
velocity of $u=0.1$ from $t=0$ to $t=1/u= 10$, which is exactly one period
through the domain.  Use $N=100$ grid points and $C=0.5$ and verify
that your method works by running the upwind method (we will call this
particular combination of method, initial condtions, N, and C the
``standard model'').  Make plots at the beginning of the evolution
($t=0$), 1/10 of a period ($t=1$) and a full period ($t=10$).  Then,
do some experiments to answer the following questions:

\begin{enumerate}

\item Experiment with the forward-time, center-space (FTCS) method.
  Is there any combination of N (varied from, say, 10 to $1{,}000$)
  and C (varied from $0.1-2$) that is stable for an entire period
  (i.e., until $t=1/u=10$)?

\item For the upwind method, if you vary C from $0.1-2$ for your
  $N=100$ top-hat model, how does the behavior vary as compared to the
  standard model?

\item Using the upwind method, $N=100$, and $C=0.5$, compare the
  behavior of the top-hat and Gaussian initial conditions after one
  period for $C=0.1,0.5,0.9,1,1.1, 1.5,2$.  How do the results differ?

\item Qualitatively, how would you describe the overall evolution in
  the solutions
  between the two methods and two initial conditions over time?
  
\end{enumerate}

Answer these questions in the \texttt{ANSWERS.md} file, and include
any plots that are needed to demonstrate your results.  Please also
include any questions you have about the reading!


% Then, do the following:

% \begin{enumerate}

% \item Implement the advection equation using the forward-time,
% centered-space (FTCS) discretization as in Exercise 4.4 in Zingale and
% $C=0.1,0.5,1.0,2.0$ for $N=100$.  Make plots at 1/10 of a period (t=1)
% and a full period (t=10).  Verify that it is unconditionally unstable
% by plotting your answer at varius times - i.e., show that oscillations
% take over and destroy the square wave.  Experiment with N and C - how
% does changing these things change the answer?

% \item Do the same thing for the first order finite-difference upwind
% method.  How is it different?

% \item Do you have any questions or things you'd like us to discuss in
% class?

% \end{enumerate}

% \noindent \textbf{Handing it in:} Include your code, your plots, and
% your answers to the questions about (in the file )
% in your assignment.

\end{document}
