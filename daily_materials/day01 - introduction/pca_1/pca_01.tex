\documentclass[10pt]{article}
 \usepackage[margin=1in]{geometry} 
\usepackage{amsmath,amsthm,amssymb,amsfonts,color, titling}
 
\setlength{\droptitle}{-20mm} 

\usepackage[colorlinks=true,linkcolor=red,urlcolor=blue]{hyperref}

\newcommand{\N}{\mathbb{N}}
\newcommand{\Z}{\mathbb{Z}}
 
\newenvironment{problem}[2][Problem]{\begin{trivlist}
\item[\hskip \labelsep {\bfseries #1}\hskip \labelsep {\bfseries #2.}]}{\end{trivlist}}
%If you want to title your bold things something different just make another thing exactly like this but replace "problem" with the name of the thing you want, like theorem or lemma or whatever
 


\title{Pre-class assignment \# 1}
\author{Brian O'Shea, \\PHY-905-005, Computational Astrophysics and
  Astrostatistics\\Spring 2023}
 \date{} % leave blank to have no date

\begin{document}
 
\maketitle

\vspace{-5mm}

\noindent
\textbf{This assignment is due by 11:59 p.m. on Monday, January 9, 2023.}

\vspace{5mm}

\noindent
\textbf{Instructions:}  Read through the materials below and follow the
instructions/answer the questions at the end.  Turn in all written answers in some readily
readable format -- preferably a text file or MarkDown, but Word or PDF
will do in a pinch.  Please turn in your homework by committing to the
Git repository you hopefully retrieved this assignment from!  If that
doesn't work, email it to me at
\href{mailto:oshea@msu.edu}{oshea@msu.edu}.  In the future, we will
exclusively use GitHub and a GitHub Classroom to hand out and turn in
all assignments.

\vspace{5mm}

\noindent
\textbf{Reading assignment:}

\begin{enumerate}

\item Class syllabus and tentative course calendar (PDFs; provided)
\item Git tutorial: \url{http://rogerdudler.github.io/git-guide/}
\item Chapter 1 and Section 2.7 of \textit{Computational Physics},
  by M. Newman
\item ``Best Practices for Scientific Computing'' by G. Wilson et al. (\href{https://arxiv.org/abs/1210.0530}{arXiv:1210.0530})
\item ``The Importance of Computation in Astronomy Education'' by M. Zingale et al. (\href{https://arxiv.org/abs/1606.02242}{arXiv:1606.02242})

\end{enumerate}



\vspace{5 mm}

\noindent
\textbf{Questions/Instructions:}

\begin{enumerate}

% \item \textbf{GitHub:} Create an account on
%   \href{https://github.com/}{GitHub} if you have not already done so and, in the writeup for this
%   assignment, tell me your GitHub username.

\item \textbf{Git tutorial:} Bookmark the Git tutorial.  We will be
  using this quite a bit in the coming weeks!

\item \textbf{MatterMost:} You already have an account on the Physics
  \& Astronomy MatterMost server, and have been added to the class
  MatterMost channel.  Please say hello on that channel (or post an
  animated gif; just indicate you know how to use it)!   We're going to
  be using MatterMost quite a bit for course discussion and announcement

  % accept the invitation to
  % \href{https://astromsu.slack.com/}{astromsu.slack.com} if you have
  % not already registered, create an account, join the channel
  % \textbf{\#compastro-fall18s}, and say something (or post an
  % animated gif; just indicate you know how to use it).  We're going to
  % be using Slack quite a bit for course discussion and announcement.

\item \textbf{Markdown:} I will distribute many documents using the
  \href{https://en.wikipedia.org/wiki/Markdown}{Markdown} syntax.  You may
  wish to install a Markdown editor.  A good one for Macs is
  \href{http://macdown.uranusjr.com/}{MacDown}, and for Linux is
  \href{https://github.com/retext-project/retext}{ReText}.  If you are a
  Windows user I don't have any advice on this, but I'm sure the
  Internet does.

\item Read the course syllabus and tentative course calendar, and
  write down any questions that you might have about those, or any
  subjects that you think are missing from the calendar.
  \textbf{Also}, write down one or more things that you would like to
  leave this class being able to do that you do not yet know how to
  do!

\item Read the Newman chapter+section and the Wilson paper.  For each of the two
  separately, (1) provide a brief (1-2 paragraph) summary of what you
  feel the main points of the reading are, and (2) list two or more
  questions that occurred to you as you were reading it. 

\item Read, but do not summarize, the Zingale paper.  I am a co-author
  on that paper, and it describes my overall feeling regarding
  computational education in astronomy and astrophysics.

\end{enumerate}


\end{document}