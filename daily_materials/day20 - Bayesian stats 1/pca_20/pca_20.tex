\documentclass[10pt]{article}
 \usepackage[margin=1in]{geometry} 
\usepackage{amsmath,amsthm,amssymb,amsfonts,color, titling}
 \usepackage{listings}

\setlength{\droptitle}{-20mm} 

\usepackage[colorlinks=true,linkcolor=red,urlcolor=blue]{hyperref}

\title{Pre-class assignment \#20}
\author{PHY-905-005\\Computational Astrophysics and
  Astrostatistics\\Spring 2023}
 \date{} % leave blank to have no date

\begin{document}
 
\maketitle

\vspace{-10mm}

\noindent
\textbf{This assignment is due before class on Monday April 17, 2023.}
 Turn in all materials via GitHub.

\vspace{5mm}

\noindent
\textbf{Reading:}

\begin{enumerate}

\item Watch \href{https://www.youtube.com/watch?v=9TDjifpGj-k}{this
    short video on Bayesian statistics} -- this is a very good and
  concise tutorial! 
  
\item Chapters 1-5.3 in
  \href{http://www.springer.com/us/book/9783319152868}{Bayesian
    Methods for the Physical Sciences}, by Andreon et al.  (available
  electronically 
  at the
  \href{https://link-springer-com.proxy2.cl.msu.edu/book/10.1007\%2F978-3-319-15287-5}{MSU
    Library} and via the included PDF).  Note that it's 5 chapters,
  but each one is VERY short.

\item Jake VanderPlas' blog post on
  \href{http://jakevdp.github.io/blog/2014/03/11/frequentism-and-bayesianism-a-practical-intro/}{Frequentist
    vs. Bayesian statistics} (note: this blog post was converted into
  a paper \href{https://arxiv.org/abs/1411.5018}{on the arXiv}.).
  This reading provides a complementary approach to Andreon's.

\end{enumerate}

\noindent
\textbf{Your assignment:}

\begin{enumerate}

\item In Bayes' Theorem (Equation 2.7 and following in Andreon), what
  is the utility of the inclusion of prior and evidence probabilities?
  (i.e., the rightmost term and term in the denominator in equation
  2.7).  In other words, why is it useful to include them?

\item Fundamentally, what is the difference between the frequentist
  and Bayesian approaches toward statistical inference?  How does this
  manifest in the way that these approaches are used?

\item In the file \texttt{ANSWERS.md}, write down any questions that
you have about the material you read,
 any points that are not clear, or anything you'd
like to know more about.  Aim for at
least 3 questions/unclear points/etc. 

\end{enumerate}

\noindent \textbf{Handing it in:} Include 
your answers to the questions (in the file \texttt{ANSWERS.md})
in your assignment.


\end{document}
