\documentclass[10pt]{article}
 \usepackage[margin=1in]{geometry} 
\usepackage{amsmath,amsthm,amssymb,amsfonts,color, titling,graphicx}
 \usepackage{listings}

\setlength{\droptitle}{-20mm} 

\usepackage[colorlinks=true,linkcolor=red,urlcolor=blue]{hyperref}

\title{In-class assignment \# 10}
\author{Brian O'Shea, \\PHY-905-005, Computational Astrophysics and
  Astrostatistics\\Spring 2023}
 \date{} % leave blank to have no date

\begin{document}
 
\maketitle

\vspace{-5mm}


\noindent \textbf{Instructions:}  In today's class we are going to do
two things:

\begin{enumerate}

\item Verify that the 2$^{nd}$ order finite-volume advection code you
wrote in the pre-class assignment works correctly for the various
conditions we've described -- left- and right-traveling top hat and
Gaussian boundary conditions, with both periodic and Dirichlet
boundary conditions.  Test each of the four sets of initial conditions
with periodic boundary conditions for 1 full period, and do the same
for one of the sets of boundary conditions for the Dirichlet BCs and
verify that the wave disappears when it hits the boundaries.

\item Implement the \texttt{minmod} flux limiter (equations 5.11 and
5.12) in Zingale into your code.  First spend some time discussing it
with your group, and then implement it and test it for the velocities
and initial conditions described above (but just use periodic boundary
conditions), running for 1 period.  Do you see the same results shown
in Figures 5.4 and 5.10 of Zingale's notes?

\end{enumerate}

\vspace{5mm}
\noindent
As per usual, submit your code, plots, etc. via GitHub!

\end{document}