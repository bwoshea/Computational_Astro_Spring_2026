\documentclass[10pt]{article}
 \usepackage[margin=1in]{geometry} 
\usepackage{amsmath,amsthm,amssymb,amsfonts,color, titling}
 \usepackage{listings}

\setlength{\droptitle}{-20mm} 

\usepackage[colorlinks=true,linkcolor=red,urlcolor=blue]{hyperref}

\title{Pre-class assignment \# 10}
\author{Brian O'Shea, \\PHY-905-005, Computational Astrophysics and
  Astrostatistics\\Spring 2023}
 \date{} % leave blank to have no date

\begin{document}
 
\maketitle

\vspace{-5mm}

\noindent
\textbf{This assignment is due the evening of Monday Feb. 13, 2023.}
 Turn in all materials via GitHub.

\vspace{5mm}

\noindent
\textbf{Reading:}

\begin{enumerate}

\item Chapter 5 (``Second- (and Higher-) Order Advection'') of Mike Zingale's
\href{http://bender.astro.sunysb.edu/hydro_by_example/CompHydroTutorial.pdf}{Computational
Hydrodynamics Tutorial}, which you can find in the pre-class assignment for Day 9.
Focus on Sections 5.1-5.4, but make sure to read Section 5.5 as well and get a
general sense of the motivations and tradeoffs with higher-order
methods.

\item \href{https://en.wikipedia.org/wiki/Flux_limiter}{Wikipedia page
    on Flux Limiters}

\end{enumerate}

\noindent
\textbf{Your assignment:}  


\begin{enumerate}


  
\item Based on your linear wave advection code from the previous
class, implement the 2$^{nd}$ order finite-volume advection in one
dimension that is discussed in Sections 5.1 and 5.2 of Zingale's
lecture notes.  Don't implement the minmod slope limiter (described in
Section 5.2.1), but think about how you would do so since we'll do it
in class!  Unlike in the pre-class assignment from the previous class,
\textbf{make sure to include ghost zones in your NumPy arrays so you
can set arbitrary boundary conditions.}  To do so, add the capability
to your code to add an arbitrary number of ghost zones on each end of
the arrays of values (1 or 2 is generally what we'll need), and also
add a function to take care of boundary conditions.  Have options for
periodic and Dirichlet (constant-value) boundary conditions.
\textbf{Make sure that you test both sets of boundary conditions, as
well as left- and right-traveling waves} (i.e., $u > 0$ and $u < 0$).
Periodic boundary conditions should behave as they did before; with
the Dirichlet boundary condition, waves should disappear when they
reach the edges of the domain.

\item Examine the behavior of both the square wave and the Gaussian wave
after 1 and 10 periods (i.e., loops through the domain), and compare
it to the 1$^{st}$ order upwind method you implemented prior to the
last class.  Vary both the CFL condition and number of grid cells for
both the finite-volume method and the upwind method, and describe how
the behavior differs for the two.  Write a function that calculates
and returns the L2 and infinity norms between the initial conditions
and the end state (described in Section 1.2.4 of Zingale's notes), and use it to compared the two different methods
for $C=1$ as a function of grid size from $N=16$ to $N=1{,}024$ --
what do you see?  (Note: make sure that you're going the correct
number of time steps - after one period, the centers of the pulses
should be exactly on top of each other, or every norm measurement will
have a huge offset!)

\item Think about how you might implement a slope limiter into your
code for the 2$^{nd}$ order finite-volume method, but don't actually
implement it.  In \texttt{ANSWERS.md}, describe how this
implementation differs from the previous question.

\item What are at least three questions that you have from the readings
that you'd like us to address in class?
Put those questions in \texttt{ANSWERS.md}.

\end{enumerate}


\noindent \textbf{Handing it in:} Include your code, your plots, and
your answers to the questions about (in the file \texttt{ANSWERS.md})
in your assignment.

\end{document}
