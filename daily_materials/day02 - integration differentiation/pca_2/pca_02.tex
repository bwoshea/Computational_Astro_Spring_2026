\documentclass[10pt]{article}
 \usepackage[margin=1in]{geometry} 
\usepackage{amsmath,amsthm,amssymb,amsfonts,color, titling}
 \usepackage{listings}

\setlength{\droptitle}{-20mm} 

\usepackage[colorlinks=true,linkcolor=red,urlcolor=blue]{hyperref}

\title{Pre-class assignment \# 2}
\author{PHY-905-003, Computational Astrophysics and
  Astrostatistics\\Spring 2026}
 \date{} % leave blank to have no date

\begin{document}
 
\maketitle

\vspace{-5mm}

\noindent
\textbf{This assignment is due the evening of Wednesday, Jan. 14, 2026.}

\vspace{5mm}

\noindent
\textbf{Instructions:} Read the materials below and follow the
instructions/answer the questions at the end.  Turn in your code and
all materials via the GitHub Classroom - in the directory where this
assignment exists (i.e., the directory corresponding to the Git
repository that you have cloned), add whatever new files you have
created to the repository by typing ``git add mynewfile'' (you can use
wild cards as well).  Then, commit the new files and any changed files
by typing ``git commit -a'' (the ``-a'' means ``commit every change
and new thing'' -- if you typed ``git commit mynewfile'' it would just
commit that file).  Finally, push your changes to the repository by
typing ``git push''.

\vspace{5mm}

\noindent
\textbf{What to turn in:}
Turn in plots (and the Python scripts or Jupyter notebooks
used to generate them), text files, etc.  \textbf{Do not} turn in
object files, binary files, or very large files of any kind unless you
are explicitly asked to do so!


\vspace{5mm}

\noindent
\textbf{Reading:}

\begin{enumerate}

\item Chapter 5 of \textit{Computational Physics}, by M. Newman

\item ``What every computer scientist should know about floating-point
  arithmetic,'' by D. Goldberg \\
  (\href{http://dl.acm.org/citation.cfm?id=103163}{doi:10.1145/103162.103163};
  PDF provided).  Note that this is a long and math-heavy article, and
  I don't want you to read it in its entirety!  Please skip the theorems
  and focus on understanding the general idea behind floating-point
  arithmetic and its limitations (for example, the types of errors and the
  challenges that emerge from representing real numbers using binary
  arithmetic).  I suggest first reading the Wikipedia article on floating
  point arithmetic (linked below), and then reading this article -- it
  will help to contextualize it.

\item Sections 3.1 and 3.2 of \textit{An Introduction to Computational Physics},
  by T. Pang (optional reading; PDF provided) 

\item 
  \href{https://en.wikipedia.org/wiki/Floating_point}{Wikipedia article
    on floating-point numbers}  (optional reading)

\item
  \href{https://en.wikipedia.org/wiki/Numerical_differentiation}{Wikipedia
  article on numerical differentiation}  (optional reading)

\item
  \href{https://en.wikipedia.org/wiki/Numerical_integration}{Wikipedia
  article on numerical integration}  (optional reading)

\end{enumerate}

\vspace{5 mm}

\noindent
\textbf{Questions:}

\begin{enumerate}

\item Consider the function $f(x) = e^x \sin(x)$.  Calculate the
derivative $f'(x) = \frac{df}{dx}$ at $x=2$ using the three-point
formula given in Equation 5.102 of Newman (Equation 3.6 of Pang), with $h(N) =
0.5/10^N$ and N=1,2,3,4,5,6.  Note: you should do
this in a general way: write a function that takes as arguments f(x),
X (the point where you wish to evaluate $f'(x)$), and $h(N)$, and returns
the value of $f'(X)$.  Create a plot that shows $f'(X)$ as calculated
at $X=2$ for the values of N shown above.

\item Using the trapezoidal rule (Equation 5.3 in Newman or Equation 3.23 in Pang) and the same
function f(x) as above, calculate $F = \int_0^{10} f(x)dx$ using
$10^N$ equally-sized steps in the interval $x=0$ to $x=10$, with
N=1,2,3,4,5,6.  Note: write a function that takes as arguments f(x),
the starting and ending points of the interval over which you wish to
integrate, and the number of equal-sized steps you wish to take over
the interval, and returns the integrated quantity F.  Create a plot
that shows F as calculated over the given interval for the values of N
shown above.

\item 
We define $f'(X)_{an}$ to be the analytic solution to the derivative
  described in the first question, and $f'(X)^N_{num}$ to be the
  numerical solution to the integral using an interval size $h(N)$.
  The fractional error, $\epsilon$, is defined as
  $\epsilon_N = \frac{|f'(X)_{num,N} - f'(X)_{an}|}{|f'(X)_{an}|}$.  Create a
  \textbf{log-log plot} that shows $\epsilon_N$ as a function of
  $h(N)$ for the values of N shown.  If you fit a power law to it,
  approximately what is the exponent of the best-fit power law?
  (Note: I'm asking you to eyeball it, not calculate the fit precisely.)

\item After you have done the readings and completed the assignment
  above, please put at least one question that you have about each
  part of the reading (Chapter 5 of Newman and the Goldberg article)
  in the file called \texttt{ANSWERS.md}.  Please also make a note in that
  text file that says how long it took you to do this assignment!

\end{enumerate}


\end{document}