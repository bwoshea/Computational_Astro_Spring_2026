\documentclass[10pt]{article}
 \usepackage[margin=1in]{geometry} 
\usepackage{amsmath,amsthm,amssymb,amsfonts,color, titling}
 \usepackage{listings}

\setlength{\droptitle}{-20mm} 

\usepackage[colorlinks=true,linkcolor=red,urlcolor=blue]{hyperref}

\title{In-class assignment \# 2}
\author{Brian O'Shea, \\PHY-905-005, Computational Astrophysics and
  Astrostatistics\\Spring 2023}
 \date{} % leave blank to have no date

\begin{document}
 
\maketitle

\vspace{-5mm}

\noindent
\textbf{Note:} Before you get started with writing code, read the
instructions below and \textbf{make a prediction for the plot that you think
you're going to see}  Write that down in the file called
\texttt{ANSWERS.md} and then commit and push those changes to the repository.

\vspace{5mm}

\noindent 
\textbf{Instructions:} Implement the two-point, three-point,
and five-point formulae for numerical differentiation (Newman
equations 5.90 and
5.107, and Table 5.1; alternately, Pang equations
3.4, 3.6, and 3.9) in a single function, using a function argument to
decide which one you're going to use.  Have the arguments to the
function be (1) the point X where you are interested in calculating a
derivative,
 (2) a function f(x) whose derivative you wish to calculate
at point X (specifically, a software function that returns the value
of a mathematical expression!), 
(3) the grid spacing $\Delta x$, and (4) a flag showing which of
the three methods of estimating the derivative you wish to use.
Invent a somewhat complex mathematical function of your own choosing
that has a non-zero analytic first derivative that you can calculate, and use that to
verify the correctness of your implementations.

Then, at a point X and initial grid spacing $h_0$ of your choosing,
calculate the fractional error of each of these approximations for a
variety of grid spacings $h(N) = h_0 / 10^N$, $N=0, 1, 2, ..., 20$ (or
even higher).  Recall that the fractional error, $\epsilon$, is
defined as $\epsilon_N = \frac{|f'(X)_{num,N} -
f'(X)_{an}|}{|f'(X)_{an}|}$, where $f'(X)_{an}$ is the analytic
solution to the derivative at a point X and $f'(X)^N_{num}$ is the
numerical solution to the derivative using an interval size $h(N)$.
Plot the fractional error for each of the three numerical derivatives as a
function of grid spacing.  How do they behave?  Is this what you
expected?  Make sure to describe this in \texttt{ANSWERS.md}.

Finally, compare the results to your predictions.  Do they make sense?
Why do you think that you see this outcome?

\textbf{If you have time} and are inclined to do so, try to duplicate
the results from your own code using
\href{https://docs.scipy.org/doc/scipy/reference/generated/scipy.misc.derivative.html}{scipy.misc.derivative}.
Do you get the same answer?

\vspace{5mm}

\noindent 
\textbf{What to turn in:} Turn in \texttt{ANSWERS.md}, any
source code you wrote, any plots you created (and the scripts you used
to create them).  \textbf{Do not} turn in object files or
executables!

\end{document}