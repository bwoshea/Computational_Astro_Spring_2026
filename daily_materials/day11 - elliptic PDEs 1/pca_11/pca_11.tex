\documentclass[10pt]{article}
 \usepackage[margin=1in]{geometry} 
\usepackage{amsmath,amsthm,amssymb,amsfonts,color, titling}
 \usepackage{listings}

\setlength{\droptitle}{-20mm} 

\usepackage[colorlinks=true,linkcolor=red,urlcolor=blue]{hyperref}

\title{Pre-class assignment \#11}
\author{PHY-905-005\\Computational Astrophysics and
  Astrostatistics\\Spring 2023}
 \date{} % leave blank to have no date

\begin{document}
 
\maketitle

\vspace{-5mm}

\noindent
\textbf{This assignment is due the evening of Wednesday March 1, 2023.}
 Turn in all materials via GitHub.

\vspace{5mm}

\noindent
\textbf{Reading:}

\begin{enumerate}

% \item Sections 1.2.4, 2, and 3 of Mike Zingale's
%   \href{http://bender.astro.sunysb.edu/hydro_by_example/CompHydroTutorial.pdf}{Computational
%   Hydrodynamics Tutorial}  (also included in a previous pre-class assignment).  Note
% that this comes with a really fantastic
% \href{https://github.com/Open-Astrophysics-Bookshelf/numerical_exercises}{Git
%   repository} full of example codes written in Python.

\item Chapter 7.1-7.5 of of \textit{An Introduction to Computational Physics},
  by T. Pang (PDF; provided) 

\item (optional) Sections 9.1 and 9.2 of \textit{Computational Physics}, by Newman.

\end{enumerate}

\noindent
\textbf{Your assignment:}  

\begin{enumerate}

% \item After reading the assigned sections of Zingale's notes, list at
%   least two questions or points of confusion that you have in the file \texttt{ANSWERS.md}.

\item After reading the assigned sections of Pang, list at least two questions that
  you have about solving elliptical equations in the file \texttt{ANSWERS.md}.

\item Solve the problem in Section 7.4 of Pang -- the distortion of a
bench due to a person sitting on it, as described in Equations 7.64
and 7.67 -- using the matrix method.  Note that you don't need to
write your own linear algebra solver -- use the solver available in
the SciPy
\href{https://docs.scipy.org/doc/scipy/reference/linalg.html}{\texttt{linalg}}
package.  Think carefully about the boundary condition that you use
here, and how it is represented in the matrix!  In the file
\texttt{ANSWERS.md}, comment on how you might treat other boundary
conditions differently.

\end{enumerate}

\vspace{5mm}

\noindent \textbf{Handing it in:} Include your code, your plots, and
your answers to the questions about (in the file \texttt{ANSWERS.md})
in your assignment.

\end{document}
