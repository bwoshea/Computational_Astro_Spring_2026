\documentclass[10pt]{article}
 \usepackage[margin=1in]{geometry} 
\usepackage{amsmath,amsthm,amssymb,amsfonts,color, titling}
 \usepackage{listings}

\setlength{\droptitle}{-20mm} 

\usepackage[colorlinks=true,linkcolor=red,urlcolor=blue]{hyperref}

\title{In-class assignment \#11}
\author{PHY-905-005\\Computational Astrophysics and
  Astrostatistics\\Spring 2023}
 \date{} % leave blank to have no date

\begin{document}
 
\maketitle


\vspace{-5mm}

\noindent \textbf{Instructions:} Solve the same problem from the
pre-class assignment (the distortion of a bench due to a person
sitting on it, as described in equations 7.64 and 7.67 of Pang) using
the relaxation method (Section 7.5 of Pang).  For boundary conditions,
try both the the Dirichlet (set functional values) and Neumann (first
derivative) methods.
 Pick a variety of different initial guesses for $u(x)$ (A
good first guess is $u(x) = 0$ everywhere, but be creative) and see
how your estimate for $u(x)$ and the residual change for each
iteration, assuming a desired tolerance of $10^{-4}$.  By how much
does the number of iterations vary depending on your initial guess for
$u(x)$?

% REMOVED - WE DON'T GET TO THIS UNTIL THE NEXT DAY OF CLASS!xs
%If you implement Jacobi iteration vs. Gauss-Seidel iteration,
%does that affect your convergence rate to the desired tolerance?

\vspace{5mm}

\noindent \textbf{Handing it in:}   Turn in all materials via GitHub.
Include your code, plots, and anything else that you used to complete
the assignment!

\end{document}
