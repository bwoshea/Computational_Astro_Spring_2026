\documentclass[10pt]{article}
 \usepackage[margin=1in]{geometry} 
\usepackage{amsmath,amsthm,amssymb,amsfonts,color, titling}
 \usepackage{listings}

\setlength{\droptitle}{-20mm} 

\usepackage[colorlinks=true,linkcolor=red,urlcolor=blue]{hyperref}

\title{Pre-class assignment \# 15}
\author{Brian O'Shea, \\PHY-905-002, Computational Astrophysics and
  Astrostatistics\\Spring 2021}
 \date{} % leave blank to have no date

\begin{document}
 
\maketitle

\vspace{-5mm}

\noindent
\textbf{This assignment is due the evening of Tuesday March 16, 2021.}
 Turn in all materials via GitHub.

\vspace{5mm}

\noindent
\textbf{Reading:}

\begin{enumerate}

\item Chapter 8 (``Euler Equations: Numerical Methods) of Mike
Zingale's
\href{http://bender.astro.sunysb.edu/hydro_by_example/CompHydroTutorial.pdf}{Computational
Hydrodynamics Tutorial}, which you can find in the pre-class
assignment for Day 8.  Focus your attention on Sections 8.1-8.5 and
8.9.1 (``Shock tubes''), but skim Section 8.2.3 (piecewise parabolic
reconstruction) rather than diving deeply into the mathematics. Also skim the other sections in Chapter 8 to get the general
idea.

\item \href{https://en.wikipedia.org/wiki/Riemann_solver}{Wikipedia
page on Riemann solvers}.  This is a short but useful resource for
understanding the various types of Riemann solver that exist.

\end{enumerate}

\noindent
\textbf{Your assignment:}  Work through the sections listed above and
make sure you understand the math.  Answer the following questions in
\texttt{ANSWERS.md}, and be prepared to discuss those answers in
class:

\begin{enumerate}
  
\item What are the key components of a code that solves the Euler
equations for the 1D Sod Shock Tube (described in Section 8.9.1)?  How
is this similar to, or different from, the codes you've created so far
for the linear advection equation and Burgers' equation?

\item What are the benefits and drawbacks of using different methods
of reconstructing the states of cell interfaces?  Specifically,
compare and contrast piecewise constant, piecewise linear, and
piecewise parabolic reconstruction (as discussed in Sections 8.2.1-3)
in terms of accuracy, simplicity, and speed.

\item Why is there such a profusion of Riemann solvers for
computational fluid dynamics?  Why doesn't everybody use the best one?

\item What, if anything, do you want to know more about numerically
solving the Euler equations after working through this chapter?

\item What are at least two questions that you have from the readings
that you'd like us to address in class?

\end{enumerate}


\noindent \textbf{Handing it in:} Submit via GitHub as usual.  Include
anything necessary, including code, plots, and 
your answers to the questions about the readings (in the file
\texttt{ANSWERS.md}) as part of your assignment.

\end{document}
