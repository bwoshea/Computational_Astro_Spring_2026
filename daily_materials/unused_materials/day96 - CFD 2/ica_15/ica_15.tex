\documentclass[10pt]{article}
 \usepackage[margin=1in]{geometry} 
\usepackage{amsmath,amsthm,amssymb,amsfonts,color, titling,graphicx}
 \usepackage{listings}

\setlength{\droptitle}{-20mm} 

\usepackage[colorlinks=true,linkcolor=red,urlcolor=blue]{hyperref}

\title{In-class assignment \# 15}
\author{Brian O'Shea, \\PHY-905-002, Computational Astrophysics and
  Astrostatistics\\Spring 2021}
 \date{} % leave blank to have no date

\begin{document}
 
\maketitle

\vspace{-5mm}


\noindent \textbf{Instructions:}  In today's class we are going to
work with the \texttt{pyro2} code
(\href{https://github.com/python-hydro/pyro2}{repository on GitHub};
\href{https://pyro2.readthedocs.io/en/latest/index.html}{documentation}),
which you have already downloaded and installed.  This is a
Python 
code written by Michael Zingale (and documented in
\href{https://ui.adsabs.harvard.edu/abs/2014A%26C.....6...52Z/abstract}{Zingale
(2014)}) which is intended to teach computational hydrodynamics,
including the Euler equations.  We are going to be experimenting
specifically with the
\href{https://pyro2.readthedocs.io/en/latest/compressible_basics.html}{compressible
  solvers}.  Please consult the
\href{https://pyro2.readthedocs.io/en/latest/running.html}{``Running''}
documentation for instructions on how to run the code, and the
\href{https://pyro2.readthedocs.io/en/latest/analysis.html}{``Analysis
  Routines''} documentation on plotting.  We're going to do the following:

\begin{enumerate}
  
\item Examine the code in the \texttt{compressible/} subdirectory,
  which implements the algorithms we've looked at in class.  Start in
  the file \texttt{compressible/simulation.py} (which is called from
  \texttt{pyro.py} in the top-level directory for \texttt{pyro2}), and
  make sure you understand what the code is doing in terms of setting
  up initial conditions, calculating the fluxes at cell faces, doing
  Riemann solves, and updating the solution.

%%%% NOTE:  USING bit.ly for the URLs below because OS X Preview does
%%%% not do sensible things with the hashtag in the URL
%%%% (see https://github.com/latex3/hyperref/issues/110; bit.ly is the workaround)
  
\item Run some test problems with the default compressible solver,
  which is a 2$^{nd}$ order finite volume method.  In particular, try the
  %\href{https://pyro2.readthedocs.io/en/latest/compressible_basics.html#sod}{Sod shock tube problem}
  \href{https://bit.ly/30KRoTK}{Sod shock tube problem}
  and the
  %\href{https://pyro2.readthedocs.io/en/latest/compressible_basics.html#sedov}{Sedov blast wave problem}.
  \href{https://bit.ly/3eKvxnK}{Sedov blast wave problem}.


  Make plots of these and look at the
  solutions compared to the analytic solutions using the comparison
  scripts (follow the directions in the linked documentation).

\item For the Sod shock tube problem, try re-running it with varied
  parameters.  See the
  %\href{https://pyro2.readthedocs.io/en/latest/compressible_basics.html#compressible-solver}{compressible solver documentation};
  \href{https://bit.ly/3lnTaU9}{compressible solver documentation};

  the general format to modify a parameter at the
  command line is to add an argument that looks like
  ``\texttt{section.option=value}'' -- i.e.,
  ``\texttt{compressible.riemann=CGF}'' to change the Riemann solver
  from the default (HLL.
  For the \texttt{compressible} solver, turn on and off the flattening, modify the artificial viscosity,
  change the limiter and Riemann solver, and change the EOS.  Use the
  analysis tools to make plots compared to the analytic solution
  (which will end up in a file named \texttt{sod\_compare.png},
  probably in the \texttt{analysis/} directory).  Rename those plots
  to something useful and submit them, along with an analysis of the
  variation.  What trends do you see?

 \item Time permitting, try doing the same thing with either the Sedov
   blast problem \textbf{or} with one of the other compressible
   solvers (\texttt{compressible\_rk}, \texttt{compressible\_fv4}, or
   \texttt{compressible\_sdc}) and their modifiable parameters.  How
   does this behave differently than the standard  2$^{nd}$ order finite volume method?

\end{enumerate}

\vspace{5mm}
\noindent
As per usual, submit your code, plots, etc. via GitHub!

\end{document}