\documentclass[10pt]{article}
 \usepackage[margin=1in]{geometry} 
\usepackage{amsmath,amsthm,amssymb,amsfonts,color, titling}
 \usepackage{listings}

\setlength{\droptitle}{-20mm} 

\usepackage[colorlinks=true,linkcolor=red,urlcolor=blue]{hyperref}

\title{Pre-class assignment \# 14}
\author{Brian O'Shea, \\PHY-905-002, Computational Astrophysics and
  Astrostatistics\\Spring 2021}
 \date{} % leave blank to have no date

\begin{document}
 
\maketitle

\vspace{-5mm}

\noindent
\textbf{This assignment is due the evening of Sunday March 14, 2021.}
 Turn in all materials via GitHub.

\vspace{5mm}

\noindent
\textbf{Reading:}

\begin{enumerate}

\item Chapter 7 (``Euler Equations: Theory'') of Mike Zingale's
\href{http://bender.astro.sunysb.edu/hydro_by_example/CompHydroTutorial.pdf}{Computational
Hydrodynamics Tutorial}, which you can find in the pre-class
assignment for Day 8.  Focus on Sections 7.1 and 7.2, and ensure
that you understand where each equation comes from using a ideal gas
(constant $\gamma$) equation of state.  Skim Section 7.3 and make sure
you get the general concept.

\item
\href{https://en.wikipedia.org/wiki/Rankine%E2%80%93Hugoniot_conditions}{Wikipedia
page on the Rankine-Hugoniot jump conditions}.  This is a really
useful resource for understanding how we derive the jump in conditions
across shocks from both first principles and the Euler equations.  

\end{enumerate}

\noindent
\textbf{Your assignment:}  Work through the derivations in Sections
7.1 and 7.2 and make sure you understand the math, using an ideal gas
law (constant $\gamma$) equation of state as needed.  Answer the
following questions in \texttt{ANSWERS.md}, and be prepared to discuss
those answers in class:

\begin{enumerate}
  
\item Why might we want to use both the conserved form and primitive
form of the Euler equations?  In other words, what might each set of
equations potentially tell us?

 \item How are the Euler equations similar to the linear advection
equation and/or Burgers' equation, and how are they different?

\item What's the point of all of the linear algebra in this
chapter?  In other words, what does it tell you about the Euler
equations and how they propagate information?

\item What are the types of waves that are in the Riemann problem
for the Euler equations?  How does your answer differ from Burgers'
equation?  (And how does this relate to the previous question, about
linear algebra?)

\item What, if anything, do you want to know more about the Euler equations after
working through this chapter?

\item What are at least two questions that you have from the readings
that you'd like us to address in class?

\end{enumerate}


\noindent \textbf{Handing it in:} Submit via GitHub as usual.  Include
anything necessary, including code, plots, and 
your answers to the questions about the readings (in the file
\texttt{ANSWERS.md}) as part of your assignment.

\end{document}
