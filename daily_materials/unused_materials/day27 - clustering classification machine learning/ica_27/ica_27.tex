\documentclass[10pt]{article}
 \usepackage[margin=1in]{geometry} 
\usepackage{amsmath,amsthm,amssymb,amsfonts,color, titling,graphicx}
 \usepackage{listings}

\setlength{\droptitle}{-20mm} 

\usepackage[colorlinks=true,linkcolor=red,urlcolor=blue]{hyperref}

\title{In-class assignment \#27}
\author{PHY-905-002, Computational Astrophysics and
  Astrostatistics\\Fall 2018}
 \date{} % leave blank to have no date

\begin{document}
 
\maketitle

\vspace{-5mm}


\noindent \textbf{Instructions:}   We're going to spend two days
working through some of
\href{https://staff.washington.edu/jakevdp/}{Jake VanderPlas'}
tutorials on
\href{https://en.wikipedia.org/wiki/Machine\_learning}{Machine
Learning}.  Jake is one of the authors of the website
\href{http://www.astroml.org/}{AstroML: Machine Learning and Data
Mining for Astronomy} as well as a very good
\href{https://www.amazon.com/Statistics-Mining-Machine-Learning-Astronomy/dp/0691151687/}{textbook
on the same subject}, and also the author of the
\href{http://shop.oreilly.com/product/0636920034919.do}{Python Data
Science Handbook}.  These tutorials use the
\href{http://scikit-learn.org/stable/}{\texttt{scikit-learn}} package, which is
a set of open-source tools for data mining and data analysis.

Before we do anything else, download Jake's
\href{https://github.com/jakevdp/sklearn\_tutorial}{scikit-learn
  tutorial}\footnote{\texttt{git clone
  https://github.com/jakevdp/sklearn\_tutorial.git}} and make sure
that you have the correct packages installed by running the notebook
\texttt{01-Preliminaries.ipynb}.  You will probably have to install
the \href{https://seaborn.pydata.org/}{Seaborn} visualization library,
which is based on matplotlib and useful for making pretty statistical
graphs.  You can do this by typing ``\texttt{pip install seaborn}'',
and you will probably have to restart your Jupyter Notebook
server after you do so to have access to Seaborn.

\vspace{5mm}

\noindent
\textbf{Useful machine learning resources:}

\begin{itemize}

\item \href{http://www.astroml.org/}{AstroML: Machine Learning and Data
Mining for Astronomy} website -- lots of helpful resources and tutorials!

\item \href{http://scikit-learn.org/stable/}{\texttt{scikit-learn}}
  website -- broadly used machine learning toolkit; lots of resources.

\item Book:
  \href{https://www.amazon.com/Statistics-Mining-Machine-Learning-Astronomy/dp/0691151687/}{Statistics,
    Data Mining, and Machine Learning in Astronomy: A Practical Python
    Guide for the Analysis of Survey Data} by Ivezic et al.

\item Book: \href{http://shop.oreilly.com/product/0636920034919.do}{Python Data
Science Handbook} by Jake VanderPlas.
(Also download \href{https://github.com/jakevdp/PythonDataScienceHandbook}{the
  Jupyter Handbook version of the entire book, c/o Jake V.})

\item A Coursera
  \href{https://www.coursera.org/learn/machine-learning}{course on
    machine learning}, taught by
  \href{http://www.andrewng.org/}{Andrew Ng} of Stanford.  This course
  was heavily recommended to me by my CMSE colleagues whose expertise
  is machine learning.

\item
  \href{https://github.com/LocalGroupAstrostatistics2015}{Astronomical
    Machine
    Learning Tutorial} from the 2015 \href{http://dept.astro.lsa.umich.edu/lgastrostatistics/}{Local Group
    Astrostatistics workshop} at the University of Michigan.  (This is
less complete than the other references listed here, but has the
advantage of brevity.)

\item \href{https://www.lsstcorporation.org/fellowship\_program}{LSST
    Data Science Fellowship} -- A two-year fellowship that is open to grad students; you can apply
  for this next year!  Even if you don't do the fellowship the
  \href{https://github.com/LSSTC-DSFP/LSSTC-DSFP-Sessions}{training
    session materials are available on GitHub}.

\end{itemize}

\end{document}