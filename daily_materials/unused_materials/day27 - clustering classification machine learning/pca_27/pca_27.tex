\documentclass[10pt]{article}
 \usepackage[margin=1in]{geometry} 
\usepackage{amsmath,amsthm,amssymb,amsfonts,color, titling}
 \usepackage{listings}

\setlength{\droptitle}{-20mm} 

\usepackage[colorlinks=true,linkcolor=red,urlcolor=blue]{hyperref}

\title{Pre-class assignment \#27}
\author{PHY-905-002\\Computational Astrophysics and
  Astrostatistics\\Fall 2018}
 \date{} % leave blank to have no date

\begin{document}
 
\maketitle

\vspace{-10mm}

\noindent
\textbf{This assignment is due the evening of Monday December 3, 2018.}
 Turn in all materials via GitHub.

\vspace{5mm}

\noindent
\textbf{Reading:}

\begin{enumerate}

\item Chapter 9 (Clustering, classification and data mining) of
  \href{http://astrostatistics.psu.edu/MSMA/}{Modern  Statistical
    Methods for Astronomy} by Feigelson and Babu

\end{enumerate}

\noindent
\textbf{Your assignment:}

\begin{enumerate}

\item Work through the tasks in the included Jupyter notebook, and
answer the questions in the notebook.

\item In the file \texttt{ANSWERS.md}, write down any questions that
you have about the material you read or the work you did in the
Jupyter notebook, any points that are not clear, or anything you'd
like to know more about.  Aim for at
least 3 questions/unclear points/etc. 

\end{enumerate}

\noindent \textbf{Handing it in:} Include your modified notebook and
your answers to the questions (in the file \texttt{ANSWERS.md})
in your assignment.

\vspace{5mm}

\noindent
\textbf{Some potentially useful Python packages:}  This will probably be helpful
for you in the future!

\begin{enumerate}

\item \href{http://www.astroml.org/}{AstroML: Machine Learning and
    Data Mining for Astronomy} (website).  

\item
  \href{https://github.com/astroML/astroML/tree/master/examples}{AstroML
  examples}.  

\item \href{http://scikit-learn.org/stable/}{scikit-learn: machine
    learning in Python}.  

\end{enumerate}

\end{document}
