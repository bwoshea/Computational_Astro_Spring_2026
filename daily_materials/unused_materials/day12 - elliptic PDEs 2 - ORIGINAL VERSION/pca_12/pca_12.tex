\documentclass[10pt]{article}
 \usepackage[margin=1in]{geometry} 
\usepackage{amsmath,amsthm,amssymb,amsfonts,color, titling}
 \usepackage{listings}

\setlength{\droptitle}{-20mm} 

\usepackage[colorlinks=true,linkcolor=red,urlcolor=blue]{hyperref}

\title{Pre-class assignment \#12}
\author{PHY-905-005\\Computational Astrophysics and
  Astrostatistics\\Spring 2023}
 \date{} % leave blank to have no date

\begin{document}
 
\maketitle

\vspace{-5mm}

\noindent
\textbf{This assignment is due the evening of Tuesday Mar. 13, 2023.}
 Turn in all materials via GitHub.

\vspace{5mm}

\noindent
\textbf{Reading:}

\begin{enumerate}

\item Chapter 9 of Mike Zingale's
  \href{http://bender.astro.sunysb.edu/hydro_by_example/CompHydroTutorial.pdf}{Computational
    Hydrodynamics Tutorial}  (PDF in a prior pre-class assignment)

\item Section 9.2 of \textit{Computational Physics}, by Newman.

\end{enumerate}

\noindent
\textbf{Your assignment:}  

\begin{enumerate}

\item After reading the assigned sections from Newman and Zingale's
notes, list at least three questions or points of confusion that you have in the file
\texttt{ANSWERS.md}.

\item Implement one of the relaxation methods described in Section 9.3
of Zingale's notes to solve the 2D Poisson equation for an $N \times
N$ grid of user-specified size, which can also accept an arbitrary
density distribution in the grid.  Make sure to appropriately include
boundary conditions, which adds two cells in each dimension.  Write
the code in a modular way -- have a function that generates the grid,
one that sets the density field in a user-specified way, one that
implements the boundary conditions, one that does the iteration, and
one that measures the residual error (as defined by Equation 9.38 of
Zingale).  Test to make sure that your solver is working correctly by
inserting a single point-mass at the center (i.e., density in a single
cell), and plot the residual, L$_2$, and L$_\infty$ norms as a
function of iteration (measure L$_2$ and L$_\infty$ vs. the analytic
solution for a point mass).
%\textbf{We will be using this in class,
%so make sure that your code works!}

\end{enumerate}

\vspace{5mm}

\noindent \textbf{Handing it in:} Include your code, your plots, and
your answers to the questions about (in the file \texttt{ANSWERS.md})
in your assignment.

\end{document}
