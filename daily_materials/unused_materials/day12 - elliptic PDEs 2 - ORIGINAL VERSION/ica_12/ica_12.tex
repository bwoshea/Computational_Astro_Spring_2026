\documentclass[10pt]{article}
 \usepackage[margin=1in]{geometry} 
\usepackage{amsmath,amsthm,amssymb,amsfonts,color, titling}
 \usepackage{listings}

\setlength{\droptitle}{-20mm} 

\usepackage[colorlinks=true,linkcolor=red,urlcolor=blue]{hyperref}

\title{In-class assignment \#12}
\author{PHY-905-005\\Computational Astrophysics and
  Astrostatistics\\Spring 2023}
 \date{} % leave blank to have no date

\begin{document}
 
\maketitle


\vspace{-5mm}

\noindent \textbf{Instructions:} Today we're going to think through
how to write a geometric multigrid solver (as described by Section 9.4
of Zingale's lecture notes), and work through an example of it from
his \href{https://github.com/python-hydro/hydro_examples}{example
codes}.

Before you do anything else, spend some time talking to the people
in your discussion group, and work through how you would come up with a modular
multigrid code.  For simplicity, you can assume that the
\textit{active region} of your 1-d grid is composed of 2$^N$ cells,
with one guard cell.  Use a text file or shared whiteboard in Zoom to create the pseudo-code
for this problem!

Then, clone Zingale's
\href{https://github.com/python-hydro/hydro_examples}{repository of example
codes}.   The multigrid example is in the subdirectory
\texttt{multigrid/}.  The files are well-commented and their purpose
is easy to understand from the comment blocks at the top of the file, but the ones that
are particularly imporant here are:

\begin{itemize}

\item \texttt{multigrid.py}  -- The file containing the multigrid
  class for cell-centered data, which implements the V-cycle method.

\item \texttt{patch1d.py} -- A class for 1-d cell-centered data, which
  performs data management, prolongation, and restriction.  It is used
  by \texttt{multigrid.py}

\item \texttt{mg\_test.py} -- A driver for the multigrid solver that
  calls \texttt{multigrid.py}, and which sets up and solves a Poisson
  problem and makes some plots of its behavior.

\end{itemize}

Before you do anything else, look through \texttt{patch1d.py} and
\texttt{multigrid.py}, and identify the routines that correspond to
what you've sketched out as pseudo-code.  Are you missing anything?
What does it do, and what equations in Section 9.4 does it correspond
to?

Once you've identified the routines, examine the files
\texttt{mg\_test.py} and \texttt{mg\_converge.py} to make sure that
you understand what they do, and how they use \texttt{multigrid.py}.
Run them, and examine the results.

Finally, we're going to examine the way that filling in the guard
cells (i.e., the boundary
conditions) can affect the results, by modifying the routine
\texttt{fill\_BC} in the file \texttt{patch1d.py}.  Modify this so
that it incorrectly fills in the boundary conditions, and then examine
the outputs of  \texttt{mg\_converge.py} again.  How has it changed?

\vspace{5mm}

\noindent \textbf{Handing it in:}   Turn in all materials via GitHub.
Include your code, plots, and anything else that you used to complete
the assignment!

\end{document}
