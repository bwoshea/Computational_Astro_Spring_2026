\documentclass[10pt]{article}
 \usepackage[margin=1in]{geometry} 
\usepackage{amsmath,amsthm,amssymb,amsfonts,color, titling}
 \usepackage{listings}

\setlength{\droptitle}{-20mm} 

\usepackage[colorlinks=true,linkcolor=red,urlcolor=blue]{hyperref}

\title{Pre-class assignment \#14}
\author{PHY-905-005\\Computational Astrophysics and
  Astrostatistics\\Spring 2023}
 \date{} % leave blank to have no date

\begin{document}
 
\maketitle

\vspace{-5mm}

\noindent
\textbf{This assignment is due the evening of Sunday March 21, 2021.}
 Turn in all materials via the GitHub Classroom.

\vspace{5mm}

\noindent
\textbf{Reading:}

\begin{enumerate}

\item Chapter 7 of \textit{Computational Physics}, by M. Newman
  \textbf{(required)} -- read Sections 7.1, 7.2, and 7.4, but focus on
  general conceptual understanding rather than  diving into the
  details regarding derivations.

\item
  \href{http://practicalcryptography.com/miscellaneous/machine-learning/intuitive-guide-discrete-fourier-transform/}{An
    intuitive guide to the discrete Fourier transform}
  \textbf{(required)} -- short but helps to build intuition.

\item
  \href{https://docs.scipy.org/doc/scipy/tutorial/fft.html}{\texttt{scipy.fft}
    tutorial} \textbf{(required)}

\item 
  \href{https://docs.scipy.org/doc/numpy/reference/routines.fft.html}{NumPy
  FFT routines}  (reference)

\item \href{http://www.jezzamon.com/fourier/index.html}{An Interactive
    Introduction to Fourier Transforms} (optional, but highly
  recommended -- it's useful
    for building intuition!)

\item
  \href{http://jakevdp.github.io/blog/2013/08/28/understanding-the-fft/}{Understanding
    the FFT algorithm} (optional, but useful for building
    intuition)

  

%\item  Sections 6.1-6.4  of \textit{An Introduction to Computational Physics},
%  by T. Pang (PDF provided; optional)


\end{enumerate}

\noindent
\textbf{Your assignment:}  After you read the appropriate chapter of
Wilson and the ``Intuitive guide to the discrete Fourier transform,''
work through the examples in the \texttt{scipy.fft} tutorial and
use those to examine the spectral properties of the functions defined
in the file \texttt{functions.py} (included with this assignment).
Please note that we're going to use this code in class, so you may
wish to make sure that your code is well-commented!

\begin{enumerate}

\item Plot both the function \texttt{fx} and \texttt{fx\_noisy} as a
function of time (you may have to zoom in on a small window of time to
see what's going on).

\item Calculate the 1D FFT of both functions for $N=10,000$.  Does the
result in frequency space like what you expected for these two
functions?  Why or why not?  And, how are the results for the two
functions different?

\item Calculate the 1D FFT of both functions for $N=5000, 2000, 1000,
500, 250, 100$.  As you decrease the number of samples, what happens?
Given what you know about how a discrete Fourier transform is
calculated, why do you think you see this behavior?

\item Do you have any questions or things you'd like us to discuss in
class?

\end{enumerate}

\noindent
\textbf{Handing it in:}
Include your code, your plots, and your answers to the questions about
FFTs (in the file \texttt{ANSWERS.md}) in your assignment.

\end{document}
