\documentclass[10pt]{article}
 \usepackage[margin=1in]{geometry} 
\usepackage{amsmath,amsthm,amssymb,amsfonts,color, titling,graphicx}
 \usepackage{listings}

\setlength{\droptitle}{-20mm} 

\usepackage[colorlinks=true,linkcolor=red,urlcolor=blue]{hyperref}

\title{In-class assignment \#13}
\author{PHY-905-005\\Computational Astrophysics and
  Astrostatistics\\Spring 2023}
 \date{} % leave blank to have no date

\begin{document}
 
\maketitle

\vspace{-5mm}


\noindent \textbf{Instructions:} In today's class we are going to
experiment with the backward-Euler implicit discretization of the 1D
diffusion equation with constant constant of diffusivity in a
\textit{finite-volume grid} and test it using the same Gaussian pulse
as in the pre-class assignment (i.e., Equation 10.14 in Zingale with
constants as in exercise 10.2).  The equation we're going to solve is:

\begin{equation}
\frac{\partial \phi}{\partial t} = k \frac{\partial^2 \phi}{\partial^2 x}
\end{equation}

\noindent Do the following:

\begin{enumerate}

\item Analytically calculate the matrix coefficients for a
backward-Euler implicit discretization assuming (i) Dirichlet boundary
conditions with $\phi_L = \phi_R = 1$ and (ii) homogeneous Neumann
boundary conditions at both ends of the grid.  Compare with your group
members to make sure you've
done it correctly!

\item Implement the backward-Euler solution for the Dirichlet boundary
condition version using the matrix method for a grid of arbitrary size
N$_{grid}$ and Courant factor C.  (Hint: can you recycle any code from
a previous pre-class or in-class assignment?)

\item Test your solution for a variety of values of the Courant factor
(C) and grid resolution (N$_{grid}$), from $C < 1$ to $C \gg 1$ and
grid sizes where the initial pulse is poorly resolved to extremely
well resolved (N$_{grid} \simeq 10$ to N$_{grid} \gg 100$).  What combinations of N$_{grid}$ and C produce good
results?  Poor results?  How large can C be and still reproduce the
expected analytic solution for a given N$_{grid}$?

\end{enumerate}

\noindent 
If you still have time after you're done with this, think about how
you might implement this problem using the Crank-Nicholson method
rather than backward-Euler.  How much would you have to modify your
code to do this?

\vspace{5mm}
\noindent
As per usual, submit your code, plots, etc. via GitHub!

\end{document}