\documentclass[10pt]{article}
 \usepackage[margin=1in]{geometry} 
\usepackage{amsmath,amsthm,amssymb,amsfonts,color, titling}
 \usepackage{listings}

\setlength{\droptitle}{-20mm} 

\usepackage[colorlinks=true,linkcolor=red,urlcolor=blue]{hyperref}

\title{Pre-class assignment \#8}
\author{PHY-905-005\\Computational Astrophysics and
  Astrostatistics\\Spring 2023}
 \date{} % leave blank to have no date

\begin{document}
 
\maketitle

\vspace{-5mm}

\noindent
\textbf{This assignment is due the evening of Monday, February 6, 2023}
 Turn in your code
and all materials via the GitHub Classroom.

\vspace{5mm}

\noindent
\textbf{Reading:}

\begin{enumerate}


\item Section 8.6 of  \textit{Computational Physics}, by Newman. 

\item  (optional)
  \href{https://en.wikipedia.org/wiki/Shooting_method}{Wikipedia article
    on the Shooting method} 

\item (optional)
  \href{https://en.wikipedia.org/wiki/Boundary_value_problem}{Wikipedia article
    on boundary value problems}  

\item (optional) Methods of solving initial value ODEs: Sections 4.6-4.9  of \textit{An Introduction to Computational Physics},
  by T. Pang (PDF; provided)

\end{enumerate}

\noindent
\textbf{Your assignment:}

\vspace{3mm}
\noindent
Consider the equation

\begin{equation}
y'' = 4y
\end{equation}

over the interval of $[0,2]$.  It has the boundary conditions $y(0) =
5$ and $y'(2) = 218.282706$.  Implement your own version of the
Shooting Method described in Section 8.6.1 of Wilson (Section 4.7 of Pang), and verify that the
solution over the given interval matches this expression for y:

\begin{equation} 
y(x) = 2 e^{2x} + 3 e^{-2x}
\end{equation}

Demonstrate agreement by plotting the numerical answer and the
analytic one on the same plot.  Re-use your code for the secant method
and one of your better numerical integrators (midpoint/RK2 or RK4 -
not Euler's method!) when creating this new method!

Separately, solve the problem using the
SciPy
\href{https://docs.scipy.org/doc/scipy/reference/generated/scipy.integrate.solve_bvp.html}{\texttt{integrate.solve\_bvp}}
method.  Do you get the same answer?

Put your answers and observations, as well as any remaining questions
that you might have, in the file \texttt{ANSWERS.md}.  Commit that, as
well as your code, plots, and related materials, to the repository.

\end{document}
