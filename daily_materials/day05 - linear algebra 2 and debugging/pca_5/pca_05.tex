\documentclass[10pt]{article}
 \usepackage[margin=1in]{geometry} 
\usepackage{amsmath,amsthm,amssymb,amsfonts,color, titling}
 \usepackage{listings}

\setlength{\droptitle}{-20mm} 

\usepackage[colorlinks=true,linkcolor=red,urlcolor=blue]{hyperref}

\title{Pre-class assignment \#5}
\author{PHY-905-005\\Computational Astrophysics and
  Astrostatistics\\Spring 2023}
 \date{} % leave blank to have no date

\begin{document}
 
\maketitle

\vspace{-5mm}

\noindent
\textbf{This assignment is due the evening of Wednesday January 25, 2023.}
 Turn in your code and all materials via the GitHub Classroom.

\vspace{5mm}

\begin{center}
\textit{``To err is human; to really foul things up requires
a computer.''} \\-- Bill Vaughan (columnist) \footnote{This is not
precisely the version that your professor first learned, but the
language is more
appropriate for polite company.}
\end{center}

\vspace{5mm}

\noindent
\textbf{Numerical Linear Algebra Reading:}

\begin{enumerate}

\item Sections 6.2-6.4 of \textit{Computational Physics}, by
  M. Newman.  \textbf{NOTE:}  Section 6.3 is long and complex -- just
  skim that to get the general idea, and focus on Sections 6.2 and 6.4!

\item Section 5.4 of \textit{An Introduction to Computational Physics},
  by T. Pang (optional; PDF provided)

\item 
  \href{https://docs.scipy.org/doc/numpy/reference/routines.linalg.html}{NumPy
  linear algebra routines}  (reference)

\item 
  \href{https://docs.scipy.org/doc/scipy/reference/linalg.html}{SciPy
  linear algebra routines}  (reference)

\end{enumerate}


\vspace{3mm}

\noindent
\textbf{Debugging Reading:}

\begin{enumerate}

\item ``Debugging truths'' (included PDF)

\item \href{http://www.patriciashanahan.com/debug/}{Debugging
    Strategy} - Patricia Shanahan, Association for Computing Machinery.
  Note that if this website seems to be non-functional, you can look
  at a saved version on the
  \href{https://web.archive.org/web/20210502035235/http://www.patriciashanahan.com/debug/}{Internet Archive}

\item Documentation for \href{https://docs.python.org/3/library/pdb.html}{pdb -- the
    Python debugger}.  Also read this short but useful
  \href{https://pythonconquerstheuniverse.wordpress.com/2009/09/10/debugging-in-python/}{pdb tutorial}

\end{enumerate}

\vspace{3mm}

\noindent
\textbf{Some potentially helpful debugging resources:}

\begin{enumerate}

\item A handy reference sheet of
  \href{http://web.stanford.edu/class/physics91si/2013/handouts/Pdb_Commands.pdf}{pdb
  commands}

\item \href{http://www.pythontutor.com/}{Python Tutor}, a tool that
  graphically visualizes the line-by-line execution of a piece of
  code.

\item 
  \href{https://pypi.org/project/ipdb/}{ipdb}, the Python debugger for
  \href{https://ipython.org/}{IPython}, which is a more feature-rich
  Python interpreter 

  \item
  \href{https://jupyterlab.readthedocs.io/en/stable/user/debugger.html}{JupyterLab
  Debugger} documentation for the
\href{https://jupyterlab.readthedocs.io/en/stable/}{JupyterLab}
notebook user interface.

\item \href{https://pypi.org/project/pudb/}{pudb}, the full-screen,
  console-based debugger for the true command line fanatic.

\item
  \href{https://code.visualstudio.com/docs/python/debugging}{VSCode
    Python debugger}, the Python debugger module for
  \href{https://code.visualstudio.com/}{Visual Studio Code} (which is
  a really awesome and useful integrated development environment, or IDE)


\end{enumerate}

\vspace{5mm}


\textbf{Note:} additional material is on the next page!

\newpage

\noindent
\textbf{Coding assignment:}  
\vspace{3mm}

Consider the following pair of functions:

\begin{equation}
\mathrm{f_1(x,y) = x^2 e^{-(x^2+y^2)/4}}
\end{equation}

\noindent
and

\begin{equation}
\mathrm{f_2(x,y) = 4 x^2 + 1.5 y^2}
\end{equation}

\noindent 
which can be found in the file \texttt{func.py}.

Implement the \textbf{multivariable Secant method} (not Newton's
method - calculate the derivative numerically, modifying code you've
written in the past!) to find the joint root
of f$_1$ and f$_2$ (which can be found by inspection to be at
$x=y=0$).  Choose a starting point that is near zero but not precisely
at zero (say, $x=y=1$), and move progressively further away.  How does
that impact convergence and number of iterations?  Compare this to the
\href{https://docs.scipy.org/doc/scipy-0.18.1/reference/generated/scipy.optimize.root.html#scipy.optimize.root}{\texttt{root}}
method in the
\href{https://docs.scipy.org/doc/scipy-0.18.1/reference/optimize.html}{\texttt{scipy.optimize}}
package.  How do your results compare, particularly as you experiment
with different methods in \texttt{root}?

\vspace{2mm}

\noindent \textbf{Handing it in:} 
Submit your \textit{easy-to-read and commented} code to the
repository (using the class coding standard as a template).  Also, put any remaining questions that you might have
about the linear algebra readings or this particular assignment in
the file \texttt{ANSWERS.md}.

\vspace{5mm}

\noindent
\textbf{Debugging assignment:}  

\begin{enumerate}

\item A simple but broken code has been included in the file
  \texttt{string\_reversal.py}, with a Jupyter Notebook version in the
  file \texttt{string\_reversal\_notebook.ipynb}.  Edit this code to add python break
  points and then use python or
  IPython at the command line, or the JupyterLab debugger in
  JupyterLab,
  to step through the code to identify the
  broken piece of code (following the suggestions in the code's
  comments as to what you should print out, where you should put break
  points, etc.)  When you fix the code, what happens?

\item Thinking about the two readings on debugging, what
  ideas/suggestions seem the most useful for you?  Which ones seem
  the hardest to implement in your day-to-day programming?  

\item Do you have a favorite debugging tip or trick that wasn't in the
  readings?  What about a favorite debugger that's part of an editor
  or Integrated Development Environment (IDE)?  Please share it in \texttt{ANSWERS.md}.

\item Do you have any additional questions, comments, or concerns
  about debugging?   Please share it in \texttt{ANSWERS.md}.

\end{enumerate}

%\vspace{5mm}

\noindent \textbf{Handing it in:} Include your modified code and
your answers to the questions (in the file \texttt{ANSWERS.md})
in your assignment.



\end{document}
