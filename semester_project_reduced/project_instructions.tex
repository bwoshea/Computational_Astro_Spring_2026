\documentclass[10pt]{article}
 \usepackage[margin=1in]{geometry} 
\usepackage{amsmath,amsthm,amssymb,amsfonts,color, titling}
 \usepackage{listings}

\setlength{\droptitle}{-20mm} 

\usepackage[colorlinks=true,linkcolor=red,urlcolor=blue]{hyperref}

\title{Semester project instructions}
\author{PHY-905-005\\Computational Astrophysics and
  Astrostatistics\\Spring 2023}
 \date{} % leave blank to have no date

\begin{document}
 
\maketitle

\vspace{-10mm}

\noindent \textbf{Overall plan:} The goals of the semester project in
this course are: (1) for you to \textit{broaden and/or deepen} your
understanding of the numerical techniques you have learned in this
class, (2) to apply this knowledge to your own research or another
topic of your choice in astrophysics or physics through software that
you have written to either \textbf{analyze a dataset or model a
physical situation}; and (3) to share this knowledge with the rest of
the class via a presentation and code demonstration.
You will demonstrate achievement of these goals through
several deliverables, as detailed below.  Note that most deadlines are
given as dates, with the time of the deadline being 11:59 p.m. that
day unless otherwise specified.

\textbf{In terms of project size and complexity,} this is meant to be
doable in the roughly six week time frame and take roughly as much
time as two homework assignments (roughly 20 total hours of work).  To
that end, it may be very difficult to get something working that
directly ties to your research -- in that case, you should choose a
project that is a simplified, constrained, and/or idealized version of
what you'd actually need for your research!

Also, please note that I expect you to put all of your code, analysis
scripts, data files, and so on in the directory
\texttt{source\_code\_and\_other\_files} in this Git repository, and
regularly commit changes to the repository and push them to GitHub (in
other words, use this as the working directory for your project).
This serves three key functions: (i) it allows you to keep track of
your code changes, and revert those changes as needed; (ii) it
provides a critical backup of your project and data in case of
hardware failure, and (iii) it allows me to keep up on your progress,
and follow up as necessary.  So, please commit and push your changes
early and often!  Note also that your final code must (1) be properly
formatted and commented according to the class coding standard, and (2)
must pass a \texttt{Pylint} check with no significant warnings or
errors. 

\vspace{5mm}

\noindent
\textbf{Project components and deadlines, in order of due date:}

\begin{enumerate}

\item \textbf{Project proposal, due before class on Thursday 3/24/2023}.  Write a
brief ($\simeq 1-2$~paragraph) proposal describing the project that you would like to pursue.
Briefly explain (i) the scientific motivation for the project, (ii) how
it relates to your research and/or your scientific interests, (iii) the
numerical methods that you will learn about and implement for this
project and how this goes beyond what we've learned in class in either
breadth or depth, and (iv) your intended deliverables in terms of data
analysis products or model outputs.  Turn this in
via a text or markdown file in the  \texttt{proposal} subdirectory in
this repository.  In
class the next day we'll have a brief roundtable for everybody to
explain their proposed project, and I will also follow up with written
feedback (and may ask you to change your proposal).

\item \textbf{Project update \#1, due Wednesday 4/5/2023}.  The goal for
this update is for you to have done the necessary background reading
and started implementation of the code you need for your project, but
not necessarily have a fully-working code (though what you do have
should be in \texttt{source\_code\_and\_other\_files} and committed to
the repository).  The written update go in the subdirectory
\texttt{update\_1} and should include an update on your proposed
project that details (i) the numerical methods you're using and (if
relevant) the sources you've used to learn about them, (ii) your
progress thus far, and (iii) a discussion of any unexpected challenges
you've run into and the steps you've taken to deal with these
challenges.  Note: if your project is going poorly this is a good time
to re-evaluate your project and revise your plans, which may include
reducing its scope!  We will also have a brief roundtable in class the
following day for everybody to give a quick project update,
highlighting numerical methods and challenges.

\item \textbf{Project update \#2, due Monday 4/17/2023}.  At this
point, your code should be close to complete in terms of
implementation of its core functionality, and you should be able to
verify that it behaves as expected using simple datasets (if you've
created an analysis tool) or a simple simulation setup (if you've
written a model of some sort).  Note that these simple setups may be
useful as tests!  As with the previous update, your code should be in
the \texttt{source\_code\_and\_other\_files} directory and your
written update goes in the subdirectory \texttt{update\_2}.  The
written update should describe (i) your progress thus far, (ii) what
the tool/model is intended to do and why you believe that it is
behaving as expected based on your simpl dataset/model setup, and
(iii) what remains to be done before your final submission.
\textbf{By this point, your code should also adhere to the class
coding standard and pass checks with \texttt{Pylint}.}  We'll have a
brief set of updates in class the next day.

\item \textbf{Project presentations and code demonstrations, due
    10:00 a.m. on Wednesday 5/3/2023 (the beginning of the course's final exam session)}.  Your
final presentation can use any of the standard presentation file
formats (Powerpoint, Keynote, or PDF), and should be put in the
\texttt{presentation\_demo} subdirectory.  Presentations will be a
total of 15 minutes long, including 10 minutes for your talk and 5
minutes for questions and discussion.  Make sure that your
presentation includes the scientific justification for your project
and your goals (i.e., what your code is supposed to accomplish), the
numerical methods used (including a brief explanation of how the new
methods you have learned about work!), the results that you obtained,
and any lessons that you learned about the methods or implementation
that you think would be interesting to your colleagues.  If the
project is an idealized, reduced, etc. version of something you'd do
for your research, please also explain that!  You should
also be prepared to give a brief demonstration of your code, if
necessary.  Your presentation will be evaluated on the clarity of your
visual aids as well as the quality, clarity, and delivery of your
presentation.  Please note that talk slides are due via GitHub immediately before
we start presentation.

\item \textbf{Project final code deadline, due Friday 5/5/2023}.
This is the final deadline for your code, all of which
should go in the  \texttt{final\_code} subdirectory.

\textbf{Your code} should: (1) be properly formatted and commented
according to the class coding standard, and (2) must pass a
\texttt{Pylint} check with no significant warnings or errors.
Comments at the top of the main source file must describe how to run
the code, including a listing of any additional Python packages that
must be installed (on top of a baseline Anaconda distribution running
the most recent version of Python 3) in order to get your code to
work.  Your code will be assessed on your implementation of the chosen
algorithms, the results that it produces, code
clarity/modularity/comments, and whether it runs on my Mac laptop and
on the MSU supercomputer when using the aforementioned Anaconda
distribution.  In addition to your code, please include any data files
that are required, as well as scripts used to generate plots or do
additional analysis.  (If the datasets are larger than $\simeq 10$
megabytes, please talk to me before committing them to the repository
- there are other, better solutions!)

\end{enumerate}

\vspace{3mm}

\noindent
\textbf{NOTE:} I strongly encourage you to come and talk to me at all
stages of this project -- I'm happy to brainstorm ideas, give
suggestions, and look at paper drafts and your code.  Please use my
office hours, and if you can't make it to those contact me via MatterMost
to schedule an appointment at another time!

\end{document}
