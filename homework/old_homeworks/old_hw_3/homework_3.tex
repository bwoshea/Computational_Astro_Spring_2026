\documentclass[10pt]{article}
 \usepackage[margin=1in]{geometry} 
\usepackage{amsmath,amsthm,amssymb,amsfonts,color, titling}
 \usepackage{listings}
\usepackage{gensymb}

\setlength{\droptitle}{-20mm} 

\usepackage[colorlinks=true,linkcolor=blue,urlcolor=blue]{hyperref}

\title{Homework \# 3}
\author{PHY-905-003, Computational Astrophysics and
  Astrostatistics\\Spring 2017}
 \date{} % leave blank to have no date

\begin{document}
 
\maketitle

\vspace{-5mm}

\noindent \textbf{This assignment is due on Sunday April 2, 2017.}  
Turn in all materials via GitHub.  Include your code, plots, and
answers to any questions asked in your assignment.  Your code must (1)
be easily readable, with good use of whitespace, clear variable names,
and adequate commenting (which documents design and purpose, not
mechanics) and (2) use functions to break up code into logical
components.  The solutions to individual problems should be saved in
separate, clearly-named source files or Jupyter notebooks.  Plots
should have easily readable and logical axis labels and titles, and
the source code and data used to generate the plots should be
included.  Questions should be answered in the file
\texttt{ANSWERS.md} or in a \LaTeX-created PDF document of a similar
name (e.g., \texttt{ANSWERS.pdf}).

\vspace{2mm}

\noindent
Please note that \textbf{solutions containing loops rather than Numpy
  array operations 
are not going to be graded!} The only exception is the \texttt{while}
 time loop.

\vspace{2mm}

\noindent
\textit{Hints: Read the relevant sections in Toro carefully and
consider how the equations translate into code.  This assignment
should not require a great deal of additional code beyond your working
1D hydro code.  As a rule of thumb, one equation in Toro should
correspond to one line of code in Python when you use Numpy array
operations rather than loops.}

\vspace{5mm}

\noindent {\large\textbf{Part 1:}}   
In the two classes immediately before Spring Break you implemented the HLL
Riemann solver within a 1-D MUSCL-Hancock framework to solve the
compressible Euler equations.  Now, use your previous code to
implement the HLLC Riemann solver.  The analytic background is
presented in Section 10.4 in Toro.  A practical summary of how the
pieces fit together is given in Section 10.6.  Then, do/answer the
following:

\begin{enumerate}

\item Plot the primitive quantities and compare the results at $t=0.2$ 
for the shock-tube problem (discussed in class with 
$x_0 = 0.2$,
$\rho_L = 1$,
$u_L = 0.75$,
$p_L = 1$,
$\rho_R = 0.125$,
$u_R = 0$, and
$p_R = 0.1$,
) between using the HLL and HLLC Riemann solver.
What did you expect to see and how does it compare to what you see?
How can you explain the differences, if your expectation and result 
are not in agreement?

\item Plot the primitive quantities and compare the results at $t=1$ 
between HLL and HLLC Riemann solver for 
$x_0 = 0.5$,
$\rho_L = 1.4$,
$u_L = 0$,
$p_L = 1$,
$\rho_R = 1$,
$u_R = 0$, and
$p_R = 1$.
What kind of problem/situation does this set of initial conditions correspond to?
How do the results compare to the shock-tube problem?

\item Plot the primitive quantities and compare the results at $t=1$ 
between HLL and HLLC Riemann solver for 
$x_0 = 0.5$,
$\rho_L = 1.4$,
$u_L = 0.1$,
$p_L = 1$,
$\rho_R = 1$,
$u_R = 0.1$, and
$p_R = 1$.
What kind of problem/situation does these initial conditions correspond to?
How do the results compare to the previous two problems?

\item With the insights obtained from Question 3 how can you explain
	the results of Question 1?
	Can you prove your explanation? 
	Provide appropriate initial conditions
	and plot(s) to support your statement.

\end{enumerate}

\newpage

\vspace{5mm}

\noindent {\large \textbf{Part 2:}}  
Use a separate Jupyter notebook or text file to extend your 1-D
implementation to 2-D.  A general introduction to multi-dimensional
extension is given in Chapter 16 of Toro.  Specifically look at
Sections 16.4.1 and 16.5 (p. 561) for the most important conceptual
steps regarding the MUSCL-Hancock scheme in 2-D.

\textit{Hint: Remember to update your boundary conditions and timestep
calculation (see 16.3.2) for 2-D. You also might need to use lower CFL
values, e.g. 0.6, than in 1-D for stable simulations.}

Then, do/answer the following:
\begin{enumerate}
\item In order to verify your framework, implement the
	initial conditions for the shock-tube problem in 2-D for
	\begin{itemize}
		\item a plane shock in the x-direction and
		\item a plane shock in the y-direction.
	\end{itemize}
Plot and compare the solutions for the density across the domain along
the corresponding shock direction and at different positions (for both
HLL and HLLC Riemann solver).  How do the results compare to each
other (with respect to shock direction) and to the 1-D results?

\item Implement the initial conditions for a 2-D spherical
	explosion (the Sedov-Taylor blast wave; see Toro Sections 17.1 and 17.2), 
	i.e. 
	\begin{itemize}
		\item a 2-D domain $[0,2] \times [0,2]$ with $\mathrm{Nx} = \mathrm{Ny} = 101$,
			with an
		\item inner sphere centered at $[1,1]$ with radius $R_s = 0.4$ with
			$\rho_s = 1$, $u_x = u_y = 0$ and $p_s = 1$, and an
		\item ambient medium outside the sphere with 
			$\rho_a = 0.125$, $u_x = u_y = 0$ and $p_a = 0.1$.
	\end{itemize}
Plot and compare the primitive quantities at $t=0.25$ across the
center of the domain in x-, y-, and both diagonal directions as a
function of distance from the center of the domain.  How do these
results differ from the plane shock, and why?  Also, you can create
the inner high-pressure sphere using a variety of radii and means of
smoothing the edges (i.e., mapping a circle/sphere onto a Cartesian
grid in a way that minimizes the 'stair-step' appearance of the
initial conditions).  Does changing how you initialize the inner
region affect your results either qualitatively or quantitatively?
Try varying the radius and, separately, smoothing the edges of the
over-pressured region to see what happens and describe what you see.

\end{enumerate}

\end{document}

